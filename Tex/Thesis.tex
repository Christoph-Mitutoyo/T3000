% Hinweis: Um das finale Dokument zu erstellen, als weitere Option "final" angeben
% Dies versteckt u.a. TODO-Elemente
\documentclass[12pt,final]{report}
%für finale Version
%documentclass[12pt,final]{report}
\usepackage{style}
\usepackage{dhbwTitlepage} % Eigene Titelseite verwenden

%Code-Formatierung
\usepackage{listings}
\definecolor{dkgreen}{rgb}{0,0.6,0}
\definecolor{gray}{rgb}{0.5,0.5,0.5}
\definecolor{mauve}{rgb}{0.03,0.6,0.18}
\definecolor{lgray}{rgb}{0.2,0.2,0.2}

\lstset{frame=tb,
  aboveskip=7mm,
  belowskip=7mm,
  showstringspaces=false,
  columns=flexible,
  basicstyle=\linespread{0.9}\small\ttfamily,
  numbers=none,
  numberstyle=\tiny\color{gray},
  identifierstyle=\color{lgray},
  keywordstyle=\color{blue},
  commentstyle=\color{dkgreen},
  stringstyle=\color{mauve},
  breaklines=true,
  breakatwhitespace=false,
  tabsize=3,
  postbreak=\mbox{\textcolor{blue}{$\hookrightarrow$}\space},
}


% "Metadaten"
\title{\Large{Evaluierung von Tools zum Auffinden von Undefined Behavior}}
\project{T3000}
\author{Christoph Böhringer}
\supervisor{Dipl.-Inform. (FH) Thomas Weller}
\studentNumber{3275565}
\class{TINF18-IN}
\company{Mitutoyo CTL Germany GmbH}
\date{21.06.2021}

% Initialisierungen für Abkürzungsverzeichnis
\makenoidxglossaries
\loadglsentries{Acronyms.tex}

\addbibresource{literature.bib}

\begin{document}

\pagenumbering{Roman}

\maketitle
\chapter*{Erklärung}
\label{ch:erklaerung}

Ich versichere hiermit, dass ich meine Thesis mit dem Thema
\glqq Entwicklung eines softwaregestützten kontextabhängigen Kommunikationsassistenten \grqq
\ selbstständig verfasst und keine anderen als die angegebenen Quellen und Hilfsmittel benutzt habe.

Ich versichere zudem, dass die eingereichte elektronische Fassung mit der gedruckten Fassung übereinstimmt.

\vspace{1.5cm}
Ort, Datum

\vspace{0.5cm}
\underline{\hspace{10cm}}\\
Name
\chapter*{Abstract}
\label{ch:abstract}
\addcontentsline{toc}{chapter}{\nameref{ch:abstract}}

The change from 32 to 64 bit caused an error in a software product of Mitutoyo. \newline
This thesis analyzes the cause of the error and evaluates tools which can potentially detect similar errors. \newline
The evaluated tools include: PCLint-Plus, Cppcheck, UBSan and PVS-Studio. All of these tools can detect similar errors, with Cppcheck being the overall winner of the comparison.
\tableofcontents
\listoffigures
\listoftables
\printnoidxglossary[type=\acronymtype, title={Abkürzungen}, nogroupskip]
\printnoidxglossary[type=main]
%%%%%%%%%%%%% Inhalt %%%%%%%%%%%%%

\clearpage
\pagenumbering{arabic}

\chapter{Einleitung}
\label{ch:einleitung}

Im alltäglichen Leben kommt man in verschiedenen Situationen mit gar nicht oder nur wenig bekannten Menschen in Kontakt: an der Supermarktkasse, beim Bäcker, oder man trifft eine bekannte Person in der Stadt. In solchen Situationen kann es zu einer kurzen Unterhaltung kommen: dem Small Talk. Weitere mögliche Situationen sind: Gespräche unter Schülern/Studenten, Konversationen an einem Messestand und Diskussionen nach einer Vorlesung.

Auch in der Firma führen Mitarbeiter beispielsweise an der Kaffeemaschine und in der Frühstücks- und Mittagspause kleine Unterhaltungen. Hier kennen sich die Mitarbeiter untereinander und wissen in den meisten Fällen einige Interessen des Gesprächspartners. Schwieriger wird es, wenn man das Gegenüber nicht kennt, wie zum Beispiel auf einer Messe oder einer Fortbildung. \newline
Die meisten Teilnehmer einer solchen Fortbildung kennen nur ein bis zwei weitere Teilnehmer. Um nicht durchgehend alleine zu sein muss also Kontakt zu fremden Personen aufgebaut werden. Dabei sollte auf die Interessen des Gegenübers und eventuelle Tabu- und Konfliktthemen geachtet werden.


\section{Problemstellung}
\label{sec:problemstellung}

Konkret ist, für kommunikationsschwache Menschen, in solchen Situationen der Einstieg in das Gespräch schwierig. Aus einem geringen Informationspool über das Gegenüber gilt es einen passenden Einstieg in ein Gespräch zu finden. Standardfragen, wie beispielsweise eine Bemerkung zum Wetter, können in Betracht gezogen werden, haben aber, aus gesellschaftlicher Sicht, einen Stellenwert als unkreativen Einstieg und führen nicht zwingend zu einem angeregten Gespräch.

In einem laufenden Gespräch, kann es kommunikationsschwachen Menschen aufgrund fehlender Übung an weiterführenden Fragen, Anregungen und Themen mangeln. Die Konversation gerät ins Stocken und kann abrupt abbrechen.

Das Reden mit fremden Menschen ist für Menschen mit einer Kommunikationsschwäche oft eine Stresssituation. Stress kann dazu führen, dass das Gehirn blockiert und nicht mehr leistungsfähig ist \cite[S.~79~f.]{Roese:Lernen-OHNE-Stress}. Es kommt zu einem Blackout. Gesprächsthemen und Fragen an das Gegenüber verschwinden aus dem Gedächtnis und die Konversation wird einseitig weitergeführt oder abgebrochen.

Menschliche Hilfestellungen, beispielsweise in Form eines Mentors, sind nur vor einem Gespräch verfügbar. Eine fortlaufende Unterstützung während der Konversation ist durch eine Begleitperson realisierbar, dadurch kann allerdings auch ein Dialog zwischen der Begleitperson und dem Gesprächspartner entstehen, wobei die Hilfe benötigende Person außen vor gelassen wird.

%\todo[inline]{hier steht ein todo in einer Zeile}


\section{Aufgabenstellung}
\label{sec:aufgabenstellung}

In dieser Praxisarbeit soll eine Softwarelösung entwickelt werden, welche dem Benutzer Fragen bereit stellt, um eine Konversation zu beginnen. \newline
Die Fragen sollen aufgrund von Umgebungsvariablen ausgewählt werden, damit ein passender Einstieg in ein Gespräch ermöglicht wird. Zudem können Anwender Profile von sich selbst und ihren Gegenübern erstellen. Die Einstellungen der Profile sollen auch in die Auswahl der Fragen miteinbezogen werden. \newline
Die Anwendung soll leicht zu bedienen sein und über ein übersichtliches \gls{ui} verfügen. Dies soll eine Nutzung während einer Konversation ermöglichen, ohne dass der Redefluss stark unterbrochen wird.\newline
Die Software soll ihrem Anwender im Idealfall genug theoretische Hilfestellungen bieten, damit mögliche Anwender über die Zeit eigenständig Gespräche beginnen und diese auch weiterführen können.

%Test \todo{Hier steht ein Todo mit blöder Formatierung}
\chapter{Small Talk}
\label{ch:smalltalk}

\section{Was ist Small Talk?}
\label{sec:smalltalk_was}

Als Small Talk (engl. \textit{small} \glqq{}unbedeutend, klein\grqq{} und \textit{to talk} \glqq{}sich unterhalten\grqq{}) wird ein spontanes, zufällig entstandenes, lockeres Gespräch
bezeichnet. Die häufigsten Themen beziehen sich dabei meist auf das Privatleben der Involvierten, oder auf das Geschehen um das Gespräch. Der Ton des Gesprächs ist informell.\cite{misc:wikipedia_smalltalk} \newline
Im Alltag kann es in verschiedenen Situationen zu einem Small Talk kommen:
\begin{itemize}
    \item Auf dem Weg zur Arbeit
    \item In der Mittagspause
    \item Beim Einkaufen
    \item Beim Spazierengehen
    \item An der Supermarktkasse
    \item In öffentlichen Verkehrsmitteln
    \item Auf einer Party
    \item \dots
\end{itemize}

Small Talk kann in verschiedenen Situationen, beispielsweise in einem Bewerbungsgespräch, als \glqq{}Eisbrecher\grqq{} verwendet werden. Das Führen von Small Talk hilft dabei Interesse am
Gegenüber zu zeigen. In einem beruflichen Umfeld können somit Beziehungen zu Kollegen/-innen geknüpft werden. \newline
Die Auswahl der Gesprächsthemen orientiert sich dabei daran, wie gut man das Gegenüber bereits kennt. Handelt es sich um eine/-n Freund/-in, so kann der Small Talk auch übersprungen werden
und das Gespräch kann mit einem bevorzugten Thema begonnen werden. \cite{misc:wikipedia_smalltalk}

\section{Small Talk führen}
\label{sec:smalltalk_führen}

Typische Fragen im Small Talk, wie
\begin{itemize}
    \item Wie findest du das Wetter?
    \item Wie geht es dir/deiner Familie?
\end{itemize}
können als Gesprächseinstieg dienen, sind aber nicht weiterführend und können somit die Konversation zum Stehen bringen.  \newline
Offene Fragen, also Fragen welche nicht mit \glqq{}Ja\grqq{} oder \glqq{}Nein\grqq{} beantwortet werden können, erfordern eine detailliertere Antwort vom Gegenüber und fördern somit den
Gesprächsfluss. Beispiele für Anfänge einer offenen Frage sind \glqq{}Warum\grqq{} oder \glqq{}Was\grqq{}.

Gemeinsamkeiten, wie der Ort des Gespräches, stellen eine gute Grundlage für Themen dar. Durch diese wird eine vertraute Atmosphäre geschaffen. \cite[S.~104]{Birgelen:Ich-und-der-Kunde} \newline
Während des Small Talks bietet es sich an, öfters das Thema zu wechseln. Small Talk ist ein unbeschwertes Gespräch und wird durch Themenwechsel abwechslungsreich, leicht und locker. \cite[S.~108]{Birgelen:Ich-und-der-Kunde}

Neben der allgemeinen Gesprächsführung spielen, für eine gelungene Konversation, auch Mimik, Gestik und Haltung eine Rolle. \newline
Bei der Mimik ist hierbei auf Augenkontakt zu achten, schüchterne Menschen können dabei auch auf den Mund schauen. Des Weiteren haben ein Lächeln und das Vermeiden von Grimassen einen
positiven Einfluss auf das Gespräch. \newline
Das Verwenden der Arme und Hände zur Gestikulation in einem Gespräch machen dieses lebendiger. Eine belebte Körpersprache zeigt Offenheit und Extraversion. Versteckte Handflächen
und verschränkte Arme deuten hingegen auf Verschlossenheit hin. \cite[S.~119]{Birgelen:Ich-und-der-Kunde} \newline
Eine starke und selbstbewusste Haltung zieht Aufmerksamkeit auf den/die Redner/-in. Dies kann auch das Selbstbewusstsein des/der Redners/-in steigern, was schüchternen Menschen bei der
Gesprächsführung helfen kann. Zu einer selbstbewussten Körperhaltung gehören: \cite{misc:rhetorik_selbstbewusstsein}
\begin{itemize}
    \item Fester Stand: beide Beine fest und ungefähr schulterbreit auf dem Boden
    \item Aufrechte Schultern
    \item Gerader Rücken
    \item Nach vorne gerichteter Kopf, nicht nach unten schauen
\end{itemize}
Wenn das Gegnüber eine zurückhaltende Person ist, sollte eine zu selbstbewusste Haltung vermieden werden, da sich das Gegenüber sonst unter Druck gesetzt fühlen kann und in der Konversation somit
in die Defensive gerät.

%Begrüßung in drei schritten: hallo, ich bin der/die... , ich kenne den gastgeber durch... (Köder, zusätzliche information)

%für tipss: "nice to know" kann leicht gemacht werden -> bezieht sich auf Geburtstag(Geburtsstein, Sternzeichen,...), Augenfarbe(häufigkeit, Bedeutung, ...), Name(Bedeutung, Herkunft, ...) und das ganze Zeug -> nett für ein Gespräch
\chapter{Stand der Technik}
\label{ch:sdt}

\section{Undefined Behavior}
\label{sec:ub}

\subsection{Was ist Undefined Behavior?}
\label{subsec:ub_was}

Arten von UB hier oder lieber in nem eigenen Kapitel?

\subsection{Warum wird Undefined Behavior benutzt?}
\label{subsec:ub_warum}

\subsection{}
\chapter{Requirements Engineering}
\label{ch:requirements}

Das Requirements Engineering bezieht sich in diesem Projekt auf das Erfassen von Stakeholdern und das Ermitteln von Anforderungen.

Wegen fehlenden oder falschen Anforderungen können viele Projekte nicht die gewünschten Ziele
erreichen \cite[S.~4-9]{Ebert:SystematischesReqEng}. Aus diesem Grund ist es sinnvoll zum Beginn eines Projektes Stakeholder zu erfassen,
eine Vision aufzustellen und Anforderungen an des Endprodukt zu ermitteln.


\section{Stakeholder}
\label{sec:stakeholder}

Die Stakeholder-Analyse identifiziert wichtige Anspruchsträger des Projektes.
Die nachfolge Tabelle wurde nach \cite[S.~58-60]{Ebert:SystematischesReqEng} ausgefüllt.
Eine Analyse der Verantwortung einzelner Stakeholder entfällt, da sich diese in dieser Arbeit mit der Rolle deckt. Zusätzlich arbeit jeder
Stakeholder in der gleichen Firma/Abteilung, weshalb die Spalte \glqq{}Verfügbarkeit\grqq{} ausgelassen wird.

\begin{table}[H]
    {
        \tiny
        \begin{tabularx}{\linewidth}{|X|X|X|X|X|}
            \hline
            Rolle
             & Name
             & Aufgabenbeschreibung
             & Hintergrundinformationen
             & Konfliktpotenzial
            \\
            \hline
            \cline{1-5}
            Gutachter Intern
             & TW
             & Aus- und Weiterbildung des Personals
             & Betreut/bewertet die Studienarbeit \newline
            Beratung/Analyse von \ref{sec:windbg}
             &
            \\
            \cline{1-5}
            Gutachter DHBW
             &
             & Gutachter
             & Bewertet die Studienarbeit
             &
            \\
            \cline{1-5}
            Entwickler
             &
             & Softwareentwicklung
             & Soll Tools benutzen um \gls{ub} zu vermeiden
             &
            \\
            \cline{1-5}
            Tester
             &
             & Testen der Software
             & Weniger Fehler in der Software erleichtern die Arbeit
             &
            \\
            \cline{1-5}
            Architekt
             & DB
             & Kontrollieren des Source Code falls Selbstkontrolle nicht klappt
             & Hat den Bug bearbeitet und behoben
             & Konflikt mit bisher verwendetem/vorgeschlagenem Tool
            \\
            \cline{1-5}
            Build Manager
             & TF
             & DevOps
             & Verantwortlich für TFS Zugang und Build Prozesse
             &
            \\
            \cline{1-5}
            Geschäftsführer
             & PK
             & Ressourcenverteilung (Zeit, Verfügbarkeit der Trainer, \dots)
             & Finanziert Studium \newline
            Finanziert Tool
             & Zeitliche Verfügbarkeit einer Lösung \newline
            Kosten des Tools
            \\
            \cline{1-5}
            Student
             & CB
             & Verfasst die Arbeit
             &
             &
            \\
            \hline
        \end{tabularx}
    }
    \caption{Stakeholder}
    \label{tab:stakeholder}
\end{table}
\chapter{L"osungsfindung}
\label{ch:loesungsfindung}

\section{Umgebung}
\label{sec:umgebung}

Da Strike Up portabel und während einem Gespräch verwendbar sein soll (vgl. \ref{tab:nichtfunktional} NF-1, NF-2), schränkt dies die Verfügbarkeit der möglichen Plattformen ein. \newline
Eine Softwarelösung für einen Computer/Laptop ist in diesem Fall nicht sinnvoll, da eine unauffällige Verwendung eines Laptops (oder eines Computers) während einem Gespräch nur möglich ist, wenn die betroffene Person bereits an einem Laptop sitzt. Dies ist in öffentlichen Plätzen wie Bus, Bahn, Park, Geschäftsessen oder einer Messe meist nicht der Fall.

Als portable Umgebung bieten sich daher Smartphones und Smartwatches an. \newline
In Deutschland besitzen 81\% der Bevökerung ab 14 Jahren ein Smartphone, in jungen Altersgruppen steigt der Prozentsatz auf über 95\% \cite{misc:marktforschung_smartphone}. Smartphones können innerhalb weniger Sekunden aus der Tasche geholt und gestartet werden, dies ermöglicht einen schnellen Gesprächseinstieg mit Strike Up. \newline
Für Smartphones bestehen zwei dominante Betriebssysteme: Android und i\acrshort{os}. Mit einem Marktanteil von 78,2\% ist Android in Deutschland Marktführer für Smartphonebetriebssysteme, gefolgt von i\acrshort{os} mit 21,3\%. Andere Betriebssysteme, wie Windows und Blackberry, besitzen einen Marktanteil von unter 0,5\%. \cite{misc:kantarworldpanel}. \newline
Für Android entwickelte Apps basieren auf Java oder Kotlin, während i\acrshort{os}-Apps in Swift oder Objective-C entwickelt werden. Des weiteren gibt es Tools und \glspl{sdk} mit welchen Apps entwickelt werden können, welche mit wenigen Einschränkungen auf beiden Betriebssystemen lauffähig sind:
\begin{itemize}
    \item \textbf{React Native}: React Native ist ein JavaScript Framework, welches die \gls{ui} in native (Android oder iOS spezifische) Elemente umwandelt. Die Logik bleibt dabei unverändert. Das Framework wird von Facebook, Instagram und Uber benutzt.
    \item \textbf{Xamarin}: Xamarin ermöglicht es Entwicklern eine gemeinsame Logik für Android und iOS zu schreiben. Die jeweilige UI wird allerdings in einer nativen Programmiersprache entwickelt.
    \item \textbf{Flutter}: Flutter ist ein \gls{sdk}, welches von Goolge erstellt, und im Jahr 2018 erstmals in der Version 1.0 veröffentlicht wurde. Das \gls{sdk} verwendet die, ebenso von Google entwickelte, Programmiersprache Dart. Flutter ermöglicht es UI-Komponenten zu entwickeln, welche auf beiden Betriebssystemen konsistent sind.
\end{itemize}

In Deutschland besaßen 2019 circa ein Drittel der Bevölkerung (36\%) \cite{misc:statista_smartwatches} eine Smartwatch.
Das Gerät ermöglicht Nutzern/Nutzerinnen das Lesen und Verfassen von Nachrichten (auch über Spracheingabe), die Überwachung von sportlichen Aktivitäten
(Pulsmessung, zurückgelegte Distanz, Schrittmesser, \dots) und über eine App auch das Abrufen von Karten und Planen von Routen.
Für eine optimale Nutzung sollte die Smartwatch dabei mit dem Smartphone gekoppelt sein. Auf dem Smartpone ist eine App installiert, welche die App auf der Smartwatch unterstützt.
Damit eine Kopplung möglich ist, müssen die Betriebssysteme der beiden Geräte kompatibel sein. \newline
Wie bei Smartphones ist auch der Smartwatchmarkt unter mehreren Marken wie Apple, Samsung, Huawei, Garmin \cite{misc:garmin} und fitbit \cite{misc:fitbit} aufgeteilt.
Jede Marke benutzt dabei ihr eigenes Betriebssystem.
Zum Beispiel sind Watch \gls{os} und Android Wear reduzierte Versionen der orginalen Betriebssysteme (iOS und Android). \newline
Den Großteil des Marktanteils, im ersten Quartal 2020, besitzt Apple mit 36,3\%. Darauf folgen Huawei (14,9\%), Samsung (12,4\%), Garmin (7,3\%) und fitbit (6,2\%) \cite{misc:canalys_smartwatch_marketshare}.


\section{Fragenfindung und -bewertung}
\label{sec:fragenfindungbewertung}

\subsection{Fragenfindung}
\label{subsec:fragenfindung}

Strike Up soll auf den/die Nutzer/-in spezifizierte Fragen bereitstellen, dies kann durch individuelle Personenprofile erziehlt werden (vgl. \ref{tab:funktional} F-12, F-6).
Die Profile enthalten dabei Daten wie Name, Geschlecht, und Alter. Daraus lassen sich bereits einige Gesprächsthemen generieren. So würde eine Person unter 25 Jahren eher über Videospiele
und Influencer oder bekannte Youtuber reden, als ein Person in einem Alter von über 75 Jahren. \newline
Für ein personalisiertes Gespräch sollten aber noch weitere Faktoren miteinbezogen werden, da die oben genannten Merkmale nur eine oberflächliche Beschreibung der Person ermöglichen.

Des Weiteren spielen Umgebungsvariablen eine Rolle bei der Fragenauswahl (vgl. \ref{tab:funktional} F-13). Besipiele für Umgebungsvariablen sind Ort, Jahreszeit und die Beziehung zum
Gegnüber. Beispielsweise beinhaltet eine Konversation auf einer geschäftlichen Messe andere Themen, als eine Konversation am Strand im Sommer. \newline
Umgebungsvariablen haben einen direkten Einfluss auf die Auswahl der Fragen/Hinweise und werden entweder von Strike Up generiert (Umgebungsvariablen welche mit Uhrzeit, Datum oder Einstellungen
in den Nutzerprofilen zusammenhängen) oder vom/von dem/der Nutzer/-in vor einem Gespräch ausgewählt (geschäftlich oder privat, kennt der/die Nutzer/-in das Gegenüber, \dots).

\subsection{Fragenbewertung}
\label{subsec:fragenbewertung}

Um Fragen und Hinweise während und bereits vor dem Gespräch an das Gegenüber anzupassen, könnten Fragen und Hinweise durch den/die Benutzer/-in im Voraus und im Gesprächsverlauf
bewertet werden.

Die Bewertung wird mittels einer Punktzahl von null bis 100 realisiert, wobei 100 die optimale Punktzahl darstellt. Der/die Nutzer/-in bewertet dabei auch Fragen/Hinweise für das
Gegenüber, soweit dies möglich ist und die Bewertung nicht durch das Gegenüber selbst stattfindet. In einem Gespräch wird bei Fragen/Hinweisen die Punktzahl des/der Nutzers/-in mit
der Punkzahl des Gegenübers addiert, wodurch eine maximale Punktzahl von 200 erreicht werden kann. Des Weiteren können Umgebungsvariablen, abhähngig von ihrer Stimmigkeit,
Punkte zu Fragen/Hinweisen addieren oder subtrahieren. \newline
Fragen und Hinweise werden anschließend nach ihrer Punktzahl geordnet und für die Konversation bereit gestellt. Fragen/Hinweise mit einer Gesamtbewertung von unter 100 Punkten werden nicht
angezeigt. Sollten jedoch weniger als zehn Fragen/Hinweise in einem Gespräch verfügbar sein, so werden auch Punktzahlen kleiner 100 eingebunden und der/die Nutzer/-in wird über einen
Warnhinweis informiert, dass möglicherweise Fragen angeboten werden, welche ihm/ihr oder dem Gegenüber nicht gefallen.

Eine weitere Möglichkeit zur Fragenbewertung ist das einführen von \glqq{}Tags\grqq{}. Ein Tag dient dabei als Schlagwort um Fragen/Hinweise zu kategorisieren. So hätte die Frage
\glqq{}Was ist ihr Lieblingstier?\grqq{} die Tags \glqq{}Tier\grqq{} und \glqq{}kennenlernen\grqq{}. Der/die Nutzer/-in kann in seinen/ihren Profileinstellungen aus einer Liste aller verfügbaren
Tags auswählen, ob er/sie das Tag positiv (soll im Gespräch vorkommen), neutral (egal) oder negativ (soll in einem Gespräch vermieden werden) bewertet. Nach demselben Prinzip werden
die Tags der Gesprächspartnerprofile bearbeitet. Standardmäßig werden alle Tags als neutral bewertet. Da es vorkommen kann, dass wenig Informationen über das Gegenüber bekannt sind,
kann dessen Profil und die damit verbundenen Tags auch während einer Konversation bearbeitet werden. \newline
Fragen und Hinweise besitzen auch Tags, welche für diese als passend oder unpassend bewertet werden.  Kommt ein Tag nicht in der Liste der passenden oder unpassenden Tags vor, so
wird es als neutral bewertet. Umgebungsvariablen verfügen in gleicher Weise über passende und unpassende Tags. \newline
Fragen/Hinweise werden, nach Ermittlung der Umgebungsvariablen, durch ihre eigenen Tags, die Tags der Umgebungsvariablen, die Tags des/der Nutzers/-in und die Tags des Gegenüber
bewertet. Hierzu wird die Liste aller unpassenden Tags einer/eines Frage/Hinweises mit den Listen unpassender Tags von Nutzer/-in, Umgebungsvariablen und Gegenüber verglichen. Immer,
wenn ein Tag aus der Liste der/des Frage/Hinweises mit einem Tag aus einer anderen Liste übereinstimmt, wird auf die/den aktuelle Frage/Hinweis ein Minuspunkt addiert. Dasselbe
geschieht mit den Listen der passenden Tags, hierbei wird bei einem übereinstimmenden Tag jedoch ein Pluspunkt addiert. Anschließend werden die Fragen/Hinweise in absteigender Reihenfolge nach der
erreichten Punktzahl sortiert. Fragen/Hinweise mit Null oder weniger Punkten werden aus dem Pool für das Gespräch entfernt. Sind weinger als zehn Fragen/Hinweise im Pool vorhanden, so
wird dieser zuerst mit \glqq{}neutralen\grqq{} (Bewertung entspricht Null) und anschließend mit \glqq{}negativen\grqq{} (Bewertung kleiner Null) Fragen aufgefüllt und der/die Nutzer/-in
erhält einen Warnhinweis, welcher auf möglicherweise als unpassend empfundene Fragen/Hinweise hinweist.


\section{Allgemeine Hilfestellungen}
\label{sec:allgemeine_hilfestellungen}

Aus F-1 und F-8 (\ref{tab:funktional}) geht hervor, dass Strike Up auch allgemeine Hinweise und Gesprächsthemen bereitstellen soll. Der /die Nutzer/-in kann sich mit Strike Up
über aktuelle Themen informieren und somit dieses Wissen in Gespräche einbringen. \newline
Allgemeine Hinweise können Tips für Haltung, Gestik, Mimik und Gesprächstechniken geben. Dadurch kann das Selbstbewusstsein und die Eloquenz des/der Nutzers/-in verbessert
werden.

Aktuelle Themen können mit Hilfe eines \gls{rssreader}s abgerufen werden. Dieser ist für den/die Nutzer/-in über einen Button auf der Startseite von Strike Up erreichbar. In den
Einstellungen des \gls{rssreader}s kann angegeben werden, von welchen Quellen die Themen bezogen werden sollen. Damit ist auch F-9 erfüllt. Die Auswahlmöglichkeiten sind hierbei vordefiniert,
da von dem/der durchschnittlichen Nutzer/-in nicht erwartet wird über \gls{rssfeed}s informiert zu sein. \newline
Die Themen werden mit Titel und einer Zusammenfassung angezeigt. Über einen Button wird der \gls{rssreader} aufgefordert den \gls{rssfeed} erneut abzurufen und die Daten zu aktualisieren.
Wird auf den Titel oder Zusammenfassung einer/-s Meldung/Themas gedrückt, so wird der/die Nutzer/-in zum vollständigen Artikel weitergeleitet.

Auf allgemeine Hinweise zur Gesprächsführung wurde bereits in \ref{ch:smalltalk} eingegangen. Das Realisieren dieser Hinweise innerhalb von Strike Up kann über tägliche Tipps umgesetzt werden.
Beim ersten täglichen Öffnen der Anwendung wird dem/der Nutzer/-in ein \gls{popup} angezeigt, welches einen allgemeinen Hinweis zur Gesprächsführung enthält. Innerhalb dieses \glspl{popup}
können frühere und zukünftige Hinweise aufgerufen werden.

\section{Datenspeicherung}
\label{sec:datenspeicherung}

Die von Strike Up bereitgestellten und verwalteten Daten benötigen einen Speicherort, welcher für Nutzer/-innen zugänglich ist und von Zugriffen durch Unbekannte geschützt ist.
Das Speichern der Daten kann in einer \gls{cloud} oder in dem geräteinternen Speicher des Smartphones durchgeführt werden. \newline
Wichtige Kriterien für eine Auswahl sind hierbei:
\begin{itemize}
    \item Speichervolumen
    \item Strukturierung der Daten
    \item Lese- und Schreibgeschwindigkeit
    \item Sicherheit vor Fremdzugriffen
    \item Verwaltung und Sicherung der Daten
\end{itemize}

Zur Datenspeicherung im geräteinternen Speicher gibt es Android SharedPreferences \cite{misc:sharedpreferences}. Daten werden hierbei als Key-Value-Paare in \acrshort{xml}-Dateien gespeichert.
Innerhalb der SharedPreferences einer App können Dateien erstellt, gelöscht und bearbeitet werden. Das Erstellen von neuen Ordnern und die damit verbundene Strukturierung der Daten ist jedoch nicht möglich. \newline
Mit zusätzlichen Bibliotheken wie Gson \cite{misc:gson} können Java Objekte in \gls{json} Objekte umgewandelt werden und somit als Key-Value-Paar in den Shared Preferences gespeichert
werden. \newline
SharedPreferences eignen sich jedoch nur für Datensätze unter 100KB, da die Daten im \gls{ram} gespeichert werden. Der \gls{ram} kann durch eine zu große Datenmenge überlastet werden,
wodurch die App nicht mehr wie gewollt funktioniert. Für eine größere Speicherkapazität kann der app-spezifische Speicher \cite{misc:appspecificstorage} verwendet werden. Der Unterschied
zu den SharedPreferences bestaht darin, dass Dateien  nicht im \gls{ram}, sondern im geräteinternen Speicher gespeichert werden. Zum Lesen, Schreiben und Bearbeiten der Dateien wird
deshalb ein \gls{iostream} benutzt. Eine Strukturierung der Daten durch Verzeichnisse ist auch hier nicht möglich, es wird aber die Verwendung von Datenbanken wie SQLite \cite{misc:sqlite}
ermöglicht. SQLite ist eine \gls{sql}-Datenbank und ermöglicht somit das Strukturieren der Daten in Tabellen, sowie das Verwenden aller \gls{sql}-typischen Abfragebefehle. \newline
Werden die Fragen von Strike Up im geräteinternen Speicher gespeichert, so müssen die Daten im Source Code enthalten sein und sind anschließend nur durch Updates veränderbar. Änderungen der
Daten durch den/die Nutzer/-in finden nur lokal statt und ein Austausch unter Nutzern/-innen (z.B. zur Bewertung von Fragen/Hinweisen) ist nicht möglich. \newline
Im geräteinternen Speicher gespeicherte Daten, können durch Nutzer/-innen eingesehen werden, da die Daten im Source Code enthalten sind, welcher aus der \gls{apk} extrahiert werden kann.
Für andere installierte Apps sind die Daten jedoch nicht sichtbar. \newline
Bei einer Deinstallation der App werden alle durch den/die Nutzer/-in erzeugten Daten gelöscht. Dies bedeuted, dass ein/eine Nutzer/-in von Strike Up, nach einer Deinstallation und Neuinstallation
der App, ein neues Konto erstellen muss. Zudem werden alle durch den/die Nutzer/-in erstellten Gesprächspartner/-innen gelöscht.

Datenspeicherung in einer \gls{cloud} ermöglicht die Aktualisierung der durch Strike Up bereitgestellten Fragen/Hinweisen, ohne dass die Anwendung selbst aktualisiert wird. Fragen/Hinweise,
Nutzerprofile und Profile von Gesprächspartnern/-innen werden in der \gls{cloud} gespeichert und zur Laufzeit abgerufen. \newline
Der für die \gls{cloud} benötigte Server kann durch einen privaten Server realisiert werden. Die Realisierung durch das Aufsetzen eines privaten physikalischen Servers kann jedoch aus
mehreren Gründen ausgeschlossen werden:
\begin{itemize}
    \item Die \gls{ip}-Adresse kann sich verändern
    \item Eine Authentifizierung muss erstellt werden, damit Nutzer/-innen von Strike Up auf die Daten zugreifen können
    \item Die Authentifizierung muss Zugriffe durch Fremde verhindern, um die Sicherheit persönlicher Daten zu wahren
    \item Die Hardware muss bereitgestellt und gewartet werden
    \item Strom- und Internetausfälle schalten den gesamten Server aus
\end{itemize}
Aus diesen Gründen bietet sich das Nutzen von Cloudservern anderer Anbieter wie Google, Amazon und Microsoft an. Diese (und andere) Unternehmen bieten auch fertige Cloud-Datenbanken an.

Google bietet mit Firebase \cite{misc:firebase} eine Plattform mit verschiedenen Tools, wie einer Echtzeitdatenbank, Authentifizierung, Cloud-Speicher, Crash Reports und Weiteren an.
Für Strike Up sind hierbei besonders die Echtzeitdatenbank und die Authentifizierung relevant. Die Echtzeitdatenbank speichert Daten in Form von \gls{json}-Objekten und ist somit
eine \gls{nosql}-Datenbank. Gespeicherte Daten können in der Datenbank durch Verschachtelung strukturiert werden. Die Authentifizierung ermöglicht das Anlegen von Nutzern, welche auf die
Datenbank zugreifen können. Durch Regeln kann festgelegt werden, ob Nutzer/-innen, oder Fremde, in bestimmten Bereichen der Datenbank Lese- und/oder Schreibrechte haben. Somit kann festgelegt werden,
dass nur authentifizierte Nutzer/-innen auf die gespeicherten Daten zugreifen können. Des Weiteren können die Zugriffsrechte der Nutzer/-innen auf ihre eigenen Daten beschränkt werden. \newline
Firebase besitzt ein Free-Tier, bei welchem ein Gesamtspeicher von 1GB und eine monatliche Download-Größe von 10GB zur Verfügung gestellt wird. Bei einer höheren Nutzung können die Werte
kostenpflichtig vergrößert werden.

Amazon DynamoDB \cite{misc:dynamodb} ist eine von Amazon gehostete \gls{nosql}-Datenbank, welche Daten als Key-Value-Paare in Dokumenten speichert. Wie bei Firebase ist auch hier ein
Nutzermanagement möglich. Dies geschieht über das Definieren von Zugriffsrechten für Nutzer/-innen und das Aufteilen von Nutzergruppen in Rollen (Nutzer/-in, Moderator, \dots). \newline
Im Free-Tier sind 25GB Speicher und 25 Write Capacity Units (nach Angaben von Amazon entspricht dies der Verarbeitung von ca. 200 Millionen Anfragen pro Monat) enthalten\cite{misc:amazonfreetier}.

Microsoft stellt mit Azure Cosmos DB \cite{misc:cosmosdb} eine auf App-Entwicklung spezialisierte \gls{nosql}-Datenbank bereit. Cosmos DB unterstützt fünf verschiedene Datenbankmodelle:
MonongoDB-\acrshort{api} (dokumentorientiert), Apache-Cassandra-\acrshort{api} (Key-Value), Gremlin-\acrshort{api} (Graphdatenbank) und Microsofts Document \gls{sql} und Table \acrshort{api}.
Die Datenbankstrukturen und \glspl{api} werden dabei auf die interne Struktur der Cosmos DB abgebildet \cite{misc:heise_cosmosdb}. Zugriffsrechte auf gespeicherte Daten können bis auf
die Ebene eines Key-Value-Paares individuell eingestellt werden. Nutzer/-innen erhalten dabei Ressource-Tokens, welche Zugriff auf bestimmte Teile der Datenbank gestatten. \newline
Der Free-Tarif für ein Azure Cosmos DB-Konto enthält 5GB Speicher, sowie 400RU/s (Anforderungseinheiten pro Sekunde).

%\section{Datenbanken}
%\label{sec:datenbanken}
%Das war vorher bei SdT, hat aber Wertungen und sollte deshalb eher hierher\newline
%gehört aber eigentlich eher in die lösungsbewertung 4head

%Die Schwächen relationaler Datenbanken liegen im Umgang mit großen Datenmengen, da dadurch Lese- und Schreibvorgänge deutlich %verlangsamt werden. Das Speichern von Bildern oder Dokumenten ist zudem nur schwer möglich.

%Da bei relationalen Datenbanken eine Trennung der Daten und Funktionen erfolgen muss, sind Objektdatenbanken für diesen %Anwendungszweck effizienter. Eine weitere Stärke ist eine gesteigerte Performance bei Abfragen. Die \gls{abfrsprache} und %objekteigene Funktionen umgehen aufwändigere relationale Abfragen. \newline
%Nachteile einer objektorientierten Datenbank zeigen sich in einer Performanceverschlechterung bei größeren Datenmengen.

%Die Stärken einer Graphdatenbank liegen in der Vernetzung von Daten und einer flexiblen Struktur. Komplizierte Datenabfragen lassen %sich durch diese Struktur vereinfachen und beschleunigen. Zudem ist das Speichern der Daten dem menschlichen Denken ähnlich und ist %somit leicht nachvollziehbar. Durch die Struktur lassen sich Daten zudem leicht auswerten und analysieren. \newline
%Probleme zeigen sich bei der Skalierbarkeit einer Graphdatenbank, da Graphdatenbanken auf eine Ein-Server-Architektur ausgelegt %sind. Des Weiteren gibt es keine
%einheitliche Abfragesprache.

%Dokumentenorientierte Datenbanken ermöglichen eine hohe Flexibilität im Umgang mit verschiedenen Datenstrukturen. Die flexible %Struktur ermöglicht auch beliebiges Hinzufügen und Entfernen von Daten.\newline
%Schwächen zeigen dokumentenorientierte Datenbanken beim Erstellen von Beziehungen zwischen Dokumenten. Zudem kann es durch die %unheitliche Struktur zu Problemen und Redundanzen kommen. Auch die Abfrage von Daten kann durch die flexible Struktur erschwert %werden.

%\begin{itemize}
%    \item Graphendatenbank
%    \item Relationale Datenbank
%    \item Dokumentdatenbank
%    \item Selbst programmierte Datenbank mit SharedReferences
%\end{itemize}

%Weil Graphendatenbank vermutlich nicht klappt müssen passende Fragen irgendwie anders zugeordnet werden: \newline
%-> Klappt nicht, weil Nodes in den bisher gesichteten Sprachen nicht dynamisch generiert werden können \newline
%-> Fragen und Profile haben zugeordnete Tags, durch welche passende Fragen ausgewählt werden \newline
%-> Evtl. noch andere Lösungsmöglichkeiten?

%Firebase kann auch temporäre Nutzer machen, die Daten werden für spätere "logins" gespeichert und können in die DB übernommen %werden, wenn der user einen Account erstellt

\chapter{L"osungsbewertung}
\label{ch:loesungsbewertung}

\section{Umgebung}
\label{sec:bewertung_umgebung}

Smartwatches sind internetfähig und können ohne die Unterstützung eines Smartphones auskommen. Da Smartwatches wie eine Armbanduhr am Handgelenk des/der Nutzers/-in sind, kann Strike Up schnell und unauffällig verwendet werden. Besonders während einem Gespräch ist ein kurzer Blick auf die Uhr weniger auffällig als ein Blick auf das Smartphone. \newline
Die Schwächen einer Smartwatch liegen allerdings in der geringen Display-Größe. Das Erstellen und Bearbeiten von Gesprächspartnern/-innen wird dadurch erschwert. Zudem schränkt die Displaygröße die Übersichtlichkeit der App ein. Das Verwenden einer Smartwatch zusätzlich zu einer App auf dem Smartphone wäre eine gute Lösung; jedoch müssen so zwei Apps entwickelt werden. \newline
Aus diesem Grund wird in dieser Arbeit Strike Up in Form einer App für Smartphones umgesetzt. Die Unterstützung durch eine Smartwatch ist eine mögliche Verbesserung für die Zukunft.

Damit Apps im App Store von Apple veröffentlicht werden dürfen, benötigt der/die Entwickler/-in eine Mitgliedschaft im Apple Developer Program. Diese kostet pro Jahr 99 US-Dollar \cite{misc:appledeveloper}.
Des Weiteren kann eine für i\gls{os} entwickelte Anwendung nur in einer Apple-Umgebung (Macbook, \dots) kompiliert werden. \newline
Aus diesen Gründen, und wegen dem Marktanteil von 21,3\% in Deutschland, ist das Entwickeln von Strike Up für eine reine i\gls{os}-Umgebung nicht sinnvoll. \newline

Um Apps im Google Play Store zu veröffentlichen, wird eine einmalige Registrierungsgebühr von 25 US-Dollar benötigt \cite{misc:androiddeveloper}. Eine jährliche Gebühr ist nicht vorhanden.

React Native ermöglicht zwar das Entwickeln einer Anwendung für iOS und Android, jedoch werden nicht alle plattformspezifischen \glspl{api} unterstützt. Nicht unterstützte \glspl{api} müssen in der nativen Programmiersprache erstellt werden. Es muss somit trotzdem für iOS und Android separat entwickelt werden. \cite{misc:reactnative_vs_native}\newline
Die nicht vollständige Unterstützung nativer \glspl{api} ist eine Schwachpunkt aller Frameworks, welche das gleichzeitige Entwickeln für iOS und Android ermöglichen. \newline
Xamarin bietet durch die Unterstützung von .Net und Microsoft Visual Studio die beste Entwicklungsumgebung. Jedoch mindert die geringe Popularität die Verfügbaren Hilfestellungen durch andere Entwickler/-innen \cite{misc:flutter_reactnative_xamarin}. \newline
Laut einer Umfrage von Stack Overflow ist Flutter das beliebteste der drei vorgestellten Frameworks \cite{misc:so_popularity}. Es ist jedoch auch das jüngste (Version 1.0 im Jahr 2018) und kann somit auf den geringsten Anteil an verfügbaren Hilfestellungen zurückgreifen. Auch die Unterstützung durch Bibliotheken von Dritten ist noch nicht so gut wie bei den anderen Frameworks.
\cite{misc:flutter_reactnative_xamarin}

Ein weiterer negativer Aspekt der Cross-Platform-Entwicklung sind Systemupdates von iOS oder Android. Ein Systemupdate kann neue Funktionen hinzufügen und alte Verändern oder Entfernen. Während bei nativen Plattformen darauf geachtet wird, dass alle Änderungen rückwärtskompatibel sind, müssen sich Cross-Platform-Frameworks erst an die Änderungen anpassen. Dadurch kann es zu fehlerhaftem Verhalten der App kommen. Bei der Entwicklung einer App, welche sich in Zukunft durch Verbesserungen und weitere Funktionen noch ändern kann, ist somit das Entwickeln auf einer nativen Plattform sinvoller.

Strike Up ist darauf ausgelegt den/die Nutzer/-in in und vor Gesprächen zu unterstützen. Umm diese Aufgabe optimal zu erfüllen, ist die App auf Feedback der Nutzer/-innen angewiesen. Daraus resultiert, dass Strike Up nach der Veröffentlichung nicht vollständig ist und somit weiterhin aktualisiert wird. \newline
Die Wahl des Betriebssystems fällt somit auf Android.

\section{Fragenbewertung}
\label{sec:bewertung_fragenbewertung}

Bei einer punktebasierten Fragenbewertung wird von dem/der Nutzer/-in verlangt, möglichst viele Fragen im Voraus zu bewerten. Dies ist bei einer hohen Fragenanzahl nicht sinnvoll. Des Weiteren wird für ein individuelles Gespräch auch das Gegenüber in die Auswahl der Fragen miteinbezogen. Somit muss der/die Nutzer/-in präemtiv entscheiden, ob dem Gegenüber einzelne Fragen gefallen. Da Strike Up bei der Auswahl passender Fragen helfen soll (vgl. \ref{tab:funktional}: F-12, F-13), ist die Auswahl passender Fragen durch den/die Nutzer/-in nicht zielführend.

Auch bei einer Fragenbewertung durch Tags wird von dem/der Nutzer/-in erwartet vor einem Gespräch auszuwählen was ihm/ihr und dem Gegenüber gefällt. Dies geschieht jedoch auf einer oberflächlicheren Basis. Es ist beispielsweise leichter zu sagen, ob das Gegenüber lieber klassische Musik oder Heavy Metal hört, als zu sagen wie gut dem Gegenüber die Frage \glqq{}Und wie war ihr Wochenende so?\grqq{} gefällt. \newline
Falls einige Tags nicht den Erwartungen entsprochen haben, können Tags auch während dem Gesprächsverlauf angepasst werden.

Die Auswahl der Fragen wird somit durch das Verwenden von Tags umgesetzt.


\section{Datenspeicherung}
\label{sec:bewertung_datenspeicherung}

Strike Up wird in Zukunft durch neue Fragen und Hinweise erweitert. Eine laufende Integration neuer Fragen und Hinweise ist mit dem Verwenden des geräteinternen Speichers nicht möglich.
Um dem internen Speicher neue Fragen hinzuzufügen, müsste Strike Up im Google Play Store aktualisiert werden. Aus diesem Grund wird eine Cloud-Datenbank verwendet.

Firebase ermöglicht Datenverwaltung über eine zentrale Konsole, auf welche nur der/die Besitzer/-in der Datenbank Zugriff hat. Die Konsole bietet weitere Funktionen, wie Crash-Reports,
Datennutzung und ein Dashboard mit Nutzerinformationen. \newline
Da Daten als \gls{json}-Objekte gespeichert werden, können von Strike Up generierte Objekte direkt in der Datenbank gespeichert und daraus wieder als Objekte gelesen werden. \newline
Firebase besitzt mit 1GB Speicher den geringsten frei verfügbaren Speicherplatz der vorgestellten Datenbanken.

Da Amazon DynamoDB Daten als Key-Value-Paare speichert, werden von Strike Up generierte Objekte entweder durch Code in \gls{json}-Objekte umgewandelt und anschließend gespeichert, oder
in Form von einzelnen Attributen als Key-Value-Paar gespeichert. Beim Schreiben und Lesen von Daten sind somit Zwischenschritte nötig. \newline
Mit 25GB frei verfügbaren Speicherplatz bietet Amazon DynamoDB den größten Speicherplatz der vorgestellten Datenbanken.

Microsoft Azure CosmosDB bietet dem/der Nutzer/-in die Wahl des Datenbankmodells. Die Wahl fällt für Strike Up auf eine objektorientierte Datenbank. \newline
Um Daten zu lesen und zu schreiben, wird die \acrshort{url} und das Passwort der Datenbank benötigt. Da diese Informationen im Klartext von Strike Up hinterlegt werden, kann jede/-r
Nutzer/-in auf diese zugreifen und somit freien Zugriff auf die Datenbank erhalten. \newline
Microsoft Azure CosmosDB wird somit als mögliche Option ausgeschlossen.

Die Wahl der Cloud-Datenbank fällt auf Firebase, da die Konsole eine übersichtliche Datenverwaltung und ein integriertes Nutzermanagement ermöglicht. Des Weiteren ermöglicht das vorhandene
Datenbankmodell das Speichern von Objekten.
\chapter{Umsetzung}
\label{ch:umsetzung}

\section{Datenbank}
\label{sec:datenbank}

\begin{figure}[htpb]
    \centering
    \includegraphics[width=0.85\textwidth]{db_minimiert}
    \caption{Minimierte Struktur der Datenbank}
    \label{img:db_minimiert}
\end{figure}
Daten werden in der Datenbank in Form vier verschiedener Kategorien gespeichert: \glqq{}EnvironmentFilters\grqq{} (Umgebungsvariablen), \glqq{}questions\grqq{} (Fragen),
\glqq{}tags\grqq{} (Tags) und \glqq{}users\grqq{} (Nutzer/-innen) (vgl. \ref{img:db_minimiert}).

Die Verwaltung der Datenbank erfolgt durch den Besitzer des Firebase-Projektes. Dies geschieht über eine Konsole. Um auf diese Konsole zuzugreifen, wird das entsprechende Google-Konto benötigt. \newline
Über die Konsole können in der Datenbank gespeicherte Daten beliebig geändert, gelöscht und hinzugefügt werden. Nutzerkonten können deaktiviert oder gelöscht werden. \newline
Bei einer Deaktivierung eines Nutezrkontos kann der/die entsprechende Nutzer/-in sich nicht mehr bei Strike Up anmelden. Des Weiteren kann eine Zurücksetzung des Passworts angefordert werden. \newline
Über die Konsole kann der aktuell genutzte Speicherplatz, sowie die in diesem und im vorherigen Monat heruntergeladene Datenmenge eingesehen werden. \newline
Zu den Nutzern/-innen werden (anonymisiert) das Land, in welchem sie Strike Up verwenden, das verwendete Betriebssystem, sowie die in der App verbrachte Zeit angezeigt. \newline
Außerdem können Zugriffsregeln für individuelle Abschnitte der Datenbank definiert werden.

Lese- und Schreibaufrufe werden im Code asynchron ausgeführt. Das bedeutet der Code \glqq{}läuft\grqq{} weiter, während der Lese-/Schreibaufruf im Hintergrund ausgeführt wird. Dies
verhindert das Finden von Fehlern mit einem \gls{debugger}, da Variablen, welche Daten aus der Datenbank enthalten, zur Laufzeit immer Null sind. Eine Fehlersuche im Zusammenhang mit
Leseaufrufen der Datenbank gestaltet sich somit schwierig. \newline
Wenn Daten aus der Datenbank in darauffolgendem Code benötigt werden, so muss auf die Fertigstellung des Lesevorgangs gewartet werden, da ansonsten mit Null-Objekten gearbeitet wird. Null-Objekte
würden zu Fehlern in der Anwendung führen. \newline
Ein Leseaufruf besteht aus einem \gls{listener}, welcher auf einen Knotenpunkt der Datenbank registriert wird. Beim Setzen eines \gls{listener}s (und wahlweise bei Veränderung der Daten
unter dem registrierten Knotenpunkt) wird dessen Funktion einmal ausgeführt. Um Null-Objekte zu verhindern, werden, auf einen Leseaufruf folgende, Funktionen in dem zugehörigen \gls{listener}
aufgerufen. Dies umgeht das asynchrone Ausführen der Leseaufrufe. Bei mehrern aufeinanderfolgenden Leseaufrufen, verzögert sich jedoch die Ausführung des Codes, da bei jedem Aufruf auf
dessen Fertigstellung gewartet wird. Dies kann in Extremfällen, und bei langsamem Internet, die Ausführung der App verlangsamen.


\section{Nutzerprofile}
\label{sec:profile}

Damit Nutzer/-innen Zugriff auf die von Strike Up bereitgestellten Funktionen haben, müssen sie sich zuerst registrieren. Eine Registrierung verhindert, in beschränktem Maße, dass die
von Strike Up genutzte Datenbank mit ungenutzten Nutzerprofilen gefüllt wird. Dies spart Speicherplatz, welcher im kostenlosen Vertrag beschränkt ist.
\begin{figure}[htpb]
    \centering
    \includegraphics[width=0.5\textwidth]{Sign_in}
    \caption{Anmeldung mit E-Mail-Adresse}
    \label{img:signin_email}
\end{figure}
\begin{figure}[htpb]
    \centering
    \includegraphics[width=0.5\textwidth]{sign_in_password}
    \caption{Aufforderung zur Eingabe des Passworts}
    \label{img:signin_password}
\end{figure}
Die Anmeldung erfolgt über eine E-Mail-Adresse und ein Passwort (siehe \ref{img:signin_email} und \ref{img:signin_password}). Ist die E-Mail-Adresse bereits auf ein Konto registriert, so
wird der/die Nutzer/-in aufgefordert das dazugehörige Passwort einzugeben (\ref{img:signin_password}). Wenn mit der E-Mail-Adresse ein neues Konto angelegt wird, so muss der/die Nutzer/-in
ein neues Passwort für diesen Account festlegen.

In der Datenbank werden Nutzer/-innen unter ihrer \gls{uid} gespeichert. Die \gls{uid} ist eine einzigartige, von Firebase automatisch generierte Kennung. Anhand dieser einzigartigen
Kennung wird jede/-r Nutzer/-in von Strike Up identifiziert. Wird eine Nutzeraccount gelöscht, so wird auch dessen \gls{uid} entfernt.

Abgesehen von der \gls{uid} besitzt ein Nutzerprofil folgende Attribute:
\begin{itemize}
    \item Name: Der Name des/der Nutzers/-in
    \item Email: Die E-Mail-Adresse des/der Nutzers/-in
    \item Alter: Das Alter des/der Nutzers/-in
    \item PreferredTags: Eine Liste aller Tags, welche der/die Nutzer/-in bevorzugt
    \item IgnoredTags: Eine Liste aller Tags, welche der/die Nutzer/-in vermeiden will
    \item ConversationalPartners: Eine Liste aller gespeicherten Gesprächspartner/-innen (siehe \ref{subsec:gespraechspartner}) des/der Nutzers/-in
\end{itemize}


\subsection{Tags}
\label{subsec:tags}

Tags werden in der Datenbank unter dem Knoten \glqq{}tags\grqq{} gespeichert. Ein Tag besitzt keine Attribute. Tags sind für jeden authentifizierten Nutzer einsehbar.

Nutzer/-innen wählen aus der Liste aller verfügbaren Tags aus, ob ihnen ein Tag gefällt oder nicht. Das Auswählen der bevorzugten und gemiedenen Tags geschieht dabei über zwei separate
Buttons. \newline
Tags, welche bereits in der Liste bevorzugter Tags sind, werden beim Hinzufügen neuer bevorzugter Tags nicht mehr zur Auswahl angeboten. Das Gleiche gilt für gemiedene Tags. Beim Hinzufügen
neuer bevorzutger Tags werden jedoch auch Tags angezeigt, welche in der Liste gemiedener Tags vorhanden sind. Dies gilt auch umgekehrt. \newline
\begin{figure}[htpb]
    \centering
    \includegraphics[width=0.35\textwidth]{alle_tags}
    \caption{Auswahl bevorzugter und gemiedener Tags}
    \label{img:alle_tags}
\end{figure}
Damit Tags nur in einer Liste vorkommen, und um die Auswahl der Tags zu vereinfachen, wird eine neue Funktion entwickelt, welche über mehrere Checkboxen das Aufteilen der Tags in die
entsprechenden Listen ermöglicht (\ref{img:alle_tags}). Rot steht dabei für gemiedene Tags und grün für bevorzugte Tags. Gelb markierte Tags sind in keiner der beiden Listen vorhanden und werden somit neutral bewertet. \newline
Ein Problem tut sich dabei bei der Implementation einer \gls{listactivity} in Android auf. Da mehr Objekte in der anzuzeigenden Liste sind, als auf den Bildschirm passen, verwendet Android
bereits für angezeigte Elemente verwendete \gls{container}, um noch nicht angezeigte Elemente zu laden \cite{misc:android_listview}. Das bedeutet: Wenn ein, mit einem Objekt gefüllter,
\gls{container} (durch scrollen) vom Bildschirm verschwindet, wird dieser \gls{container} benutz um das neu erschienene Element anzuzeigen. Der \gls{container} behält dabei die
Position des Hakens der Checkbox. Beim Scrollen, wird somit bei neu erschienen Tags der Haken an der Position der neu ausgeblendeten Tags gesetzt. Dies lässt sich durch das Verwenden
einer \gls{hashmap}, welche die Positionen der Haken speichert, umgehen. Dabei ist der Name eines Tags der \glqq{}Key\grqq{} und die Position des Hakens das \glqq{}Value\grqq{}. \newline
Aus Zeitgründen ist diese Funktion noch nicht fehlerfrei lauffähig. Deshalb geschieht die Auswahl bevorzugter und gemiedener Tags weiterhin über zwei separate Buttons.

Während der Entwicklung wurden Tags um eine ID erweitert; Tags werden somit unter ihrer ID gespeichert und besitzen das Attribut \glqq{}Name\grqq{}. \newline
Das Verwenden einer ID ermöglicht, dass Strike Up auf weitere Sprachen erweitert werden kann. So können Tags beispielsweise um die Attribute \glqq{}Name-en\grqq{} oder \glqq{}Name-fr\grqq{} erweitert
werden.

\subsection{Gesprächspartner/-innen}
\label{subsec:gespraechspartner}

Gesprächspartner/-innen werden als eine, zu einem/-r Nutzer/-in gehörende, Liste gespeichert. Jede/-r Nutzer/-in kann beliebig viele Gesprächspartner/-innen hinzufügen. \newline
Ein/-e Gesprächspartner/-in besitzt die Attribute:
\begin{itemize}
    \item Name: Name des/der Gesprächspartners/-in
    \item Gender: Geschlecht des/der Gesprächspartners/-in
    \item Alter: Alter des/der Gesprächspartners/-in
    \item IgnoredTags: Eine Liste aller Tags, welche der/die Gesprächspartner/-in vermeiden will
    \item PreferredTags: Eine Liste aller Tags, welche der/die Gesprächspartner/-in bevorzugt
    \item Notizen: Von dem/der Nutzer/-in verfasste Notizen zu dem/der Gesprächspartner/-in
\end{itemize}
Das Bearbeiten und Auswählen von Tags geschieht dabei gleich wie bei einem/-r Nutzer/-in.

Gesprächspartner/-innen werden um das Atttribut ID erweitert, da sie bisher unter ihrem Namen in der Datenbank gespeichert wurden. Das Speichern unter dem Namen erzeugte Probleme, wenn
der Name geändert wurde, da so zwei Objekte desselben Partnerprofils in der Datenbank hinterlegt waren. Gebraucht wird aber nur ein Objekt. Es entsanden also redundante Daten. \newline
Das Einführen einer unveränderlichen und einzigartigen ID ermöglicht eine problemfreie Änderung des Namens. \newline
Die ID wird von Firebase generiert. Zum erstellen dieser einzigartigen ID benutzt Firebase den gewünschten Speicherort innerhalb der Datenbank und das aktuelle Datum.

Nachdem das Profil eines/-r Gesprächspartners/-in erstellt ist, kann dieses bearbeitet und wieder gelöscht werden.

Strike Up besitzt individuelle Profile für den/die Nutzer/-in und für Gesprächspartner/-innen. Somit ist F-6 erfüllt (vgl. \ref{tab:funktional}).


\section{Fragen/Hinweise}
\label{sec:fragen_hinweise}

Fragen und Hinweise werden in der Datenbank unter \glqq{}questions\grqq{} gespeichert. Eine einzelne Frage ist dabei unter ihrer ID zu finden. \newline
Fragen besitzen folgende Attribute:
\begin{itemize}
    \item ID: Eine eindutige ID, unter welcher die Frage gespeichert ist
    \item tags: Eine Liste aller Tags, welche zu dieser Frage passen
    \item text: Der Text der Frage
\end{itemize}
Fragen und Hinweise können nicht von Nutzern/-innen bearbeitet werden. Das Verändern, Hinzufügen und Entfernen von Fragen obliegt somit dem/der Verwalter/-in der Datenbank. \newline
Ursprünglich war geplant, dass Nutzer/-innen sowohl Fragen als auch Tags zur Datenbank hinzufügen können, da die Datenbank besonders in den Anfängen von Strike Up noch nicht vollständig
ist. Diese Idee wurde jedoch aus Angst vor Missbrauch wieder gestrichen.

\subsection{Umgebungsvariablen}
\label{subsec:umgebungsvariablen}

Eine Umgebungsvariable wird mit folgenden Attributen in der Datenbank gespeichert:
\begin{itemize}
    \item ID: Eine eindutige ID, unter welcher die Umgebungsvariable gespeichert ist
    \item Name: Der Name der Umgebungsvariable
    \item IgnoredTags: Liste aller Tags, auf welche die Umgebungsvariable einen negativen Einfluss hat
    \item Preferred Tags: Liste aller Tags, auf welche die Umgebungsvariable einen positiven Einfluss hat
\end{itemize}
Umgebungsvariablen werden vor einem Gepräch entweder automatisch generiert oder durch den/die Nutzer/-in gesetzt. \newline
Die Umgebungsvariable Alter (jung/alt) wird automatisch generiert und der/die Nutzer/-in kann die Umgebungsvariable \glqq{}Privat\grqq{} oder \glqq{}Geschäftlich\grqq{}
auswählen. Alternativ kann die Auswahl einer Umgebungsvariable auch übersprungen werden. \newline
Das automatische Generieren und manuelle Auswählen von Umgebungsvariablen ist noch erweiterbar. Die bisherige Implementation stellt einen konzeptionellen Beweiß der Erfüllbarkeit von F-13 dar.

Im Laufe der Arbeit kam auf, dass die Eingabe eines statischen Alters bei Nutzerprofilen und Gesprächspartnern/-innen nicht sinnvoll ist, da sich dieses jährlich ändert.
Fragen und Hinweise beziehen sich auf das Alter und Strike Up setzt automatisch Umgebungsvariablen zu diesem (z.B. Jung oder Alt). Aus diesem Grund wird Alter als Profil-Attribut entfernt.
Stattdessen wird das Geburtsdatum verwendet. Aus diesem wird, bei einem Datenbankabruf des Profils, das Alter ermittelt. Somit aktualisiert sich das Alter eines Profils am Geburtstag.


\section{Gespräch}
\label{sec:gespräch}

Vor einem Gespräch wird aus der Liste aller verfügbaren Gesprächspartner/-innen ein/-e Gesprächspartner/-in ausgewählt oder alternativ ein/-e neue/-r Gesprächspartner/-in erstellt.
Anschließend kann der/die Gesprächspartner/-in noch einmal bearbeitet werden. Bevor die Konversation beginnt wählt der/die Nutzer/-in noch manuell Umgebungsvariablen aus.

Die Listen bevorzugter Tags des/der Nutzers/-in, des Gegenübers und der Umgebungsvariablen werden addiert. Doppelte (dreifache, vierfache, \dots) Tags sind dabei erlaubt. Das Gleiche passiert
mit den entsprechenden Listen gemiedener Tags. \newline
Ist die Listen fertig, so wird diese mit den entsprechenden Listen jeder verfügbaren Frage verglichen. Hierfür werden alle Tags der bevorzugten/gemiedenen Listen einzeln
miteinander verglichen. \newline
Kommt ein Tag sowohl in der Liste der bevorzugten (gemiedenen) Tags einer Frage, als auch in der kombinierten Liste bevorzugter Tags von Nutzers/-in, Gegenüber und Umgebungsvariablen vor,
so wird bei der Frage ein Punkt addiert (subtrahiert). Fragen starten dabei mit einer Punktzahl von null.

\begin{figure}[htpb]
    \centering
    \includegraphics[width=0.35\textwidth]{warning_conversation}
    \caption{Warnung, wenn weniger relevante Fragen hinzugefügt werden}
    \label{img:warning_conversation}
\end{figure}

Für die Konversation werden nur Fragen mit einer Punktzahl über null ausgewählt. Sollten somit jedoch weniger als zehn Fragen verfügbar sein, so werden auch Fragen mit einer Punktzahl von null
hinzugefügt. Sind immer noch weniger als zehn Fragen vorhanden, so werden auch negativ bewertete Fragen hinzugefügt. Der/die Nutzer/-in wird durch ein \gls{popup} auf diesen Umstand aufmerksam gemacht
(vgl. \ref{img:warning_conversation}).

Während einem Gespräch sieht die \gls{ui} wie in \ref{img:conversation} aus. Über die Buttons \glqq{}vorherige\grqq{} und \glqq{}nächste\grqq{} kann zwischen den für das Gespräch
verfügbaren Fragen gewechselt werden (F-5). Der Text der Fragen wird zentral im Bildschirm angezeigt, während andere Funktionen, wie Buttons, am Rand vorhanden sind (NF-3).
\begin{figure}[htpb]
    \centering
    \includegraphics[width=0.35\textwidth]{conversation}
    \caption{Frage während einer Konversation}
    \label{img:conversation}
\end{figure}
Das farbige Quadrat unter \glqq{}Du\grqq{} zeigt die Bewertung der Frage aus Sicht des Nutzers an. Das Quadrat unter \glqq{}Partner\grqq{} zeigt die Bewertung aus Sicht des Gegenübers
und das große Quadrat zeigt die Gesamtbewertung. Eine grüne Farbe steht dabei für eine positive, gelb für eine neutrale und rot für eine negative Bewertung.

Über \glqq{}Tabea Tingel\grqq{} wird das Profil des Gegenübers aufgerufen. Somit sind Notizen des Gegenübers während einem Gespräch verfügbar (F-7). Das Profil des/der Gesprächspartners/-in
kann somit auch erst während einer Konversation mit Informationen befüllt werden.


\section{Datenschutz}
\label{sec:datenschutz}

Strike Up soll keine personenbezogenen Daten an Dritte weitergeben (vgl. \ref{tab:nichtfunktional}, NF-4). Als \glqq{}Dritte\grqq{} ist hierbei Google in der Rolle des Verwalter der Daten,
zu verstehen. \newline
Google ist hierbei ein \glqq{}processor of [...] Customer Personal Data under the European Data Protection Legislation\grqq{} \cite{misc:firebase_terms}. Dies bedeutet, dass Google ein
Verwalter der Daten ist, diese aber nicht einsehen oder versenden darf. \newline
Da die zu Strike Up gehörende Datenbank von der EU aus bearbeitet wird, gilt die \gls{dsgvo}. Somit muss Google Daten an Behörden weiterleiten, falls dies gefordert wird.

Des Weiteren ist der Ersteller von Strike Up für das Einhalten der \gls{dsgvo} verantwortlich. Um dies zu garantieren wird beim ersten Öffnen der Anwendung ein Dialogfenster geöffnet,
welches den/die nutzer/-in auffordert zu bestätigen, dass er/sie über 18 Jahre alt ist. Außerdem willigt er/sie ein, dass alle von Strike Up verarbeiteten Daten gespeichert werden. \newline
Wird eine dieser zwei Bedienungen nicht erfüllt, so schließt sich Strike Up und der Ablauf wird beim nächsten Öffnen wiederholt.

\subsection{Schutz vor Datendiebstahl}
\label{subsec:datendiebstahl}

Damit Strike Up auf die Cloud-Datenbank zugreifen kann, wird eine \gls{url} zu dieser benötigt. Die \gls{url} ist dabei im Code der Anwendung enthalten. \newline
Diese \gls{url} ermöglicht Angreifern jedoch keinen Zugriff auf die Datenbank, da zudem noch das verbundene Google-Konto benötigt wird.

Eine weitere Angriffsmöglichkeit ist der im Quellcode hinterlegte \gls{api}-Key, welcher Strike Up bei der Datenbank registriert. Über den \gls{api}-Key können Angreifer auf alle in der
Datenbank gespeicherten Daten zugreifen. \newline
Dies wird durch das Aufstellen von Regeln verhindert. Regeln beziehen sich auf definierte Bereiche der Datenbank. Die für Strike Up definierten Regeln erlauben
jedem/-r authentifizierten Nutzer/-in das Lesen der Bereiche \glqq{}EnvironmentFilters\grqq{}, \glqq{}questions\grqq{} und \glqq{}tags\grqq{} (vgl. \ref{img:db_minimiert}). \newline
Da der Bereich \glqq{}users\grqq{} nutzerspezifische Daten enthält, ist nicht jedem/-r Nutzer/-in erlaubt alle darunter liegenden Knoten zu lesen. Jede/-r Nutzer/-in kann dabei nur auf
\glqq{}unter\grqq{} seiner/ihrer \gls{uid} gespeicherte Daten zugreifen. Der/die Nutzer/-in hat für diesen Bereich Lese- und Schreibrechte.
\chapter{Praxistests}
\label{ch.praxistests}

Strike Up soll Menschen beim Eröffnen und Führen von Gesprächen helfen (\ref{sec:vision}). Um die Wirksamkeit diser Hilfestellungen zu testen, wurden mit der entwickelten Applikation Praxistests durchgeführt. \newline
Gesprächspartner/-innen waren dabei zufällige Passanten auf der Straße und in Drogeriemärkten. \newline
Die folgenden Erkenntnisse beruhen auf empirischen Erfahrungen.

Während den Tests erwies sich besonders das Vorbereiten auf das Gespräch als hilfreich. Präparierte Gesprächsthemen halfen dabei, eine Konversation interessant zu gestalten und am Laufen zu halten. \newline
Da die Versorgung mit Gesprächsthemen durch Strike Up aus zeitlichen Gründen in dieser Arbeit nicht zureichend entwickelt wurde, wurden externe Quellen (News-Webseiten, Nachrichten im Fernesehen) hinzugezogen. Mit diesen Quellen
wurde das von Strike Up gewünschte Verhalten nach F-9 und F-10 simuliert.

Auf sich selbst und das Gegenüber angepasste Fragen erwiesen sich als vergleichsweise durchschnittlich nützlich. Fragen entstanden eher spontan und aus den bereits vorbereiteten Themn heraus. \newline
Das Vermeiden unpassender Fragen hingegen sorgte dafür, dass für sich selbst und das Gegenüber unangenehme Themen nicht angesprochen wurden. Dies verhinderte einen abrupten Abbruch der Konversation.

Eine unerwartete Entdeckung war, dass das Gegenüber, wenn es die Zuhilfenahme der App bemerkte, meist positiv mit einem Lachen reagierte. Als Grund für das Lachen wurde genannt, dass das Verwenden der App den/die Benutzer/-in sympathisch darstelle. Der/die Nutzer/-in würde eine Schwäche einsehen und versuche diese zu verbessern. \newline
Es kamen aber auch Fälle vor, bei welchen sich das Gegenüber ablenend gegenüber der App äußerte und das Gespräch somit beendete. Das Smartphone würde dabei dem/der Nutzer/-in die Konversationsführung wegnehmen und somit zu einer Verdummung führen.

Es wurde aber auch festgestellt, dass Strike Up um eine Schnellgespräch-Funktion erweitert werden sollte. \newline
Will der/die Nutzer/-in spontan ein Gespräch beginnen, so muss zuerst ein/-e Gesprächspartner/-in ausgewählt werden und anschließend Umgebungsvariablen gesetzt werden. Dieser Prozess kann, besonders beim Erstellen eines/-r neuen Gesprächspartners/-in Zeit in Anspruch nehmen. \newline
Um diesen Prozess zu umgehen, kann im Hauptbildschirm ein Button hinzugefügt werden, welcher sofort ein Gespräch startet. Die \gls{ui} des Gespräches entspricht dabei der \gls{ui} einer normalen Konversation. Das Profil des/der Gesprächspartners/-in wird während der Konversation entdeckt/erstellt.
\chapter{Zusammenfassung}
\label{ch:zusammenfassung}

Mit dieser Arbeit sollte ein kontextabhähniger Kommunikationsassistent entwickelt werden (\ref{sec:vision}). Dies wurde in Form einer Applikation namens Strike Up umgesetzt.

Strike Up ist eine für Android entwickelte App, welche dem/der Nutzer/-in ermöglicht ein eigenes Profil, sowie Profile von Gesprächspartnern/-innen, zu erstellen. Diese Profile enthalten bevorzugte und gemiedene Tags, anhand welcher während einer Konversation Fragen ausgewählt und vorgeschlagen werden. Des Weiteren haben automatisch generierte und durch den/die Nutzer/-in ausgewählte Umgebungsvariablen einen Einfluss auf die Auswahl der Fragen. \newline
Fragen beziehen sich somit auf bevorzugte und weniger bevorzugte Themen des/der Nutzers/-in und des Gegenübers, sowie auf die Umgebung der Konversation.

Um Strike Up nutzern zu können, müssen sich Nutzer/-innen mit einer E-Mail-Adresse und einem Passwort anmelden. Anschließend wird eine Altersbestätigung, sowie eine Einwilligung zur Speicherung der Daten gefordert.

Jegliche von Strike Up zur Verfügung gestellte, sowie durch den/die Nutzer/-in erstellte, Daten werden in der, von Google gehosteten, Cloud-Datenbank Firebase gespeichert. Nutzer/-innen können dabei alle Teile der Datenbank lesen, welche keine Informationen anderer Nutzer/-innen enthalten. Schreibzugriffe werden Nutzern/-innen nur bei selbst erstellten Daten (eigenes Profil, Gesprächspartner/-innen) gestattet.

Als besonders interessant haben sich die im Zuge dieser Arbeit durchgeführten Praxistests erwiesen. \newline
Die zufällig ausgewählten Gesprächspartner/-innen reagierten meist positiv auf das Verwenden einer App als Hilfestellung und fanden dies sogar sympathisch. \newline
Während einem Gespräch hat sich dabei die Vorbereitung auf das Gespräch als am hilfreichsten erwiesen. Weniger hilfreich war die Auswahl bevorzugter Fragen. Hierbei hat sich das Ausschließen unpassender Fragen als nützlicher erwiesen.
\chapter{Ausblick}
\label{ch:ausblick}

Bevor Strike Up im Google Play Store, oder firmenintern, veröffentlicht werden kann, müssen noch einige Features verbessert werden: Die Benutzeroberfläche sollte ansprechender werden
und das editieren von Tags könnte, für eine bessere Übersicht, in einem (statt zwei verschiedenen) Bildschirm stattfinden. \newline
Fragen und Hinweise können noch nicht durch den/die Nutzer/-in bewertet werden und ein Feedback am Ende eines Gespräches ist auch nicht vorhanden (vgl. \ref{tab:funktional}, F-2, F-4, F-10).
Des Weiteren können Fragen noch nicht aus der Auswahl ausgeschlossen werden (F-3). Auch die Versorgung mit aktuellen Themen, beispielsweise durch einen \gls{rssfeed}, wurde nicht umgesetzt (F-8, F-9).
Es gibt auch keine einfache Funktion, mit welcher ein/-e Nutzer/-in sein/ihr Konto und damit verbundene Daten löschen kann (F-11). Dies ist möglich, indem der Verwalter der Datenbank kontaktiert wird,
jedoch ist diese Funktion nicht in Strike Up intgriert.

In der aktuellen Form kann Strike Up noch in viele Richtungen verbessert und erweitert werden: Die Gesprächsunterstützung könnte von einer Unterstützung bei One-on-One-Gesprächen auf
eine Unterstützung während Gruppengesprächen und Diskussionen erweitert werden. \newline
Das Bewerten von Fragen kann durch ein Fragen-Rating, ähnlich wie bei der Fragenbewertung vor einem Gespräch, ermöglicht werden. \newline
Nutzer/-innen könnten, durch drücken eines Buttons, die aktuell vorgeschlagene Frage in eine Liste ignorierter Fragen setzen. Vor einem Gespräch würden, für die Konversation vorgeschlagene, Fragen
mit der Liste der ignorierten Fragen verglichen werden. Wenn eine Fragen in beiden Listen vorkommt, so würde diese nicht während dem Gespräch vorgeschlagen werden. \newline
Strike Up könnte während einer Konversation Fragen als Push-Benachichtigung anzeigen. Somit müsste das Smartphone zum Einsehen einer Frage nicht mehr entsperrt werden. \newline
Die App kann um eine Vorlese-Funktion erweitert werden. Fragen werden dem/der Nutzer/-in über Kopfhörer vorgelesen. Mit den Kopfhörerfunktionen (überspringen, zurück) könnte zwischen Fragen
gewechselt werden. \newline
Des Weiteren könnte der aktuelle Standord benutzt werden um auf Attraktionen in der Nähe hinzuweisen oder um Wetterinformationen abzurufen. Die Wetterinformationen könnten dabei über
OpenWeather \cite{misc:openweather} abgerufen werden. \newline
Strike Up kann über eine zusätzliche App für Smartwatches erweitert werden. Während einem Gespräch könnten somit Fragen und Hinweise auf der Smartwatch angezeigt werden und das Smartphone
bliebe somit in der Tasche.

\chapter{Anhang}
\label{ch:anhang}

\lstdefinelanguage{WinDbg}{
    alsoletter={., |, +, :, 0, >, !},
    keywords=[1]{.time, .exr, .sympath, .sympath+, .symfix, .symfix+, .symopt, .symopt+, |, ||, .reload, lm, !sym, k, .excr, .ecxr, .thread, .cxr},
    keywordstyle=[1]\color{dkgreen},
    keywords=[2]{0:000>},
    keywordstyle=[2]\color{gray},
    sensitive=false, % keywords are not case-sensitive
    morecomment=[l]{}, % l is for line comment
    morecomment=[s]{/*}{*/}, % s is for start and end delimiter
    morestring=[b]" % strings are enclosed in double quotes
} % 
\lstset{language=WinDbg}

\section{WinDbg Crash Analyse}
Vom betroffenen Absturz, der womöglich auf Undefined Behavior zurückzuführen ist, wurde ein Crash Dump erstellt.
Dieser Anhang beschreibt anhand einer aufgezeichneten Logdatei, welche Befehle genutzt wurden, um den Crash Dump zu analysieren.
Bei der analysierten Datei handelt es sich um einen Crash Dump mit vollständig enthaltenem Speicherinhalt. Die Datei ist ca. 1 GB groß.
\subsection{Grundlegende Prüfungen}
\begin{lstlisting}
0:000> ||
.  0 Full memory user mini dump: D:\MINIDUMP-20201217-150359.DMP
\end{lstlisting}
Dem Dateinamen nach wurde der Crash Dump am 17.12.2020 um 15:03:59 Uhr UTC geschrieben. Die Zeitangabe innerhalb des Crash Dumps bestätigt dies.

\begin{lstlisting}
0:000> .time
Debug session time: Thu Dec 17 16:04:00.000 2020 (UTC + 1:00)
System Uptime: 0 days 7:25:08.968
Process Uptime: 0 days 0:01:31.000
  Kernel time: 0 days 0:00:32.000
  User time: 0 days 0:00:48.000
\end{lstlisting}
Der Bug wurde am 27.11.2020 berichtet und am 11.12.2020 erstmals bestätigt. Beim vorliegenden Crash Dump handelt es sich also um die Reproduktion des Fehlers.

\begin{lstlisting}
0:000> |
.  0	id: 2b7c	examine	name: C:\MCOSMOSx64\EXE\GEOPAK64.exe
\end{lstlisting}
Das ausgeführte Programm stimmt mit dem des Fehlerberichts überein.

\begin{lstlisting}[language=WinDbg]
0:000> .exr -1
*** WARNING: Unable to verify checksum for MnGeom364.dll
ExceptionAddress: 00007ff995c1d48e (MnGeom364!tg_ajc_lin_int+0x000000000000014e)
   ExceptionCode: c0000005 (Access violation)
  ExceptionFlags: 00000000
NumberParameters: 2
   Parameter[0]: 0000000000000000
   Parameter[1]: 0000000000000000
Attempt to read from address 0000000000000000
\end{lstlisting}
Der Fehlercode und das fehlerhafte Modul stimmen ebenfalls mit dem des Fehlerberichts überein. Es kann also eine detailliertere Analyse durchgeführt werden.

\subsection{Symbole}
Für eine weitere Analyse ist das Vorhandensein von Symbolen erforderlich. Manche Symbole wurden zusammen mit dem Crash Dump abgelegt.
Diese Symbole sollten zuerst berücksichtigt werden und müssen zuerst in den Symbolpath aufgenommen werden.
\begin{lstlisting}[language=WinDbg]
0:000> .sympath "D:\temp\Bug 32147"
Symbol search path is: D:\temp\Bug 32147
Expanded Symbol search path is: d:\temp\bug 32147

************* Path validation summary **************
Response                         Time (ms)     Location
OK                                             D:\temp\Bug 32147
\end{lstlisting}

Leider stimmen bei den Symbolen die Zeitstempel nicht überein, so dass die Symbole nicht geladen werden können.
Die Prüfung des Zeitstempels kann jedoch ausgeschaltet werden.
\begin{lstlisting}[language=WinDbg]
0:000> .symopt+ 0x40
Symbol options are 0x30377:
  0x00000001 - SYMOPT_CASE_INSENSITIVE
  0x00000002 - SYMOPT_UNDNAME
  0x00000004 - SYMOPT_DEFERRED_LOADS
  0x00000010 - SYMOPT_LOAD_LINES
  0x00000020 - SYMOPT_OMAP_FIND_NEAREST
  0x00000040 - SYMOPT_LOAD_ANYTHING
  0x00000100 - SYMOPT_NO_UNQUALIFIED_LOADS
  0x00000200 - SYMOPT_FAIL_CRITICAL_ERRORS
  0x00010000 - SYMOPT_AUTO_PUBLICS
  0x00020000 - SYMOPT_NO_IMAGE_SEARCH
\end{lstlisting}

Weitere Symbole können vom Azure DevOps Server geladen werden
\begin{lstlisting}[language=WinDbg]
0:000> .sympath+ srv*d:\debug\symbols*\\tfs-build-2014\SymbolStore\Mitutoyo.MCOSMOS.BasicLibs_Release_NuGet
Symbol search path is: D:\temp\Bug 32147;srv*d:\debug\symbols*\\tfs-build-2014\SymbolStore\Mitutoyo.MCOSMOS.BasicLibs_Release_NuGet
Expanded Symbol search path is: d:\temp\bug 32147;srv*d:\debug\symbols*\\tfs-build-2014\symbolstore\mitutoyo.mcosmos.basiclibs_release_nuget

************* Path validation summary **************
Response                         Time (ms)     Location
OK                                             D:\temp\Bug 32147
Deferred                                       srv*d:\debug\symbols*\\tfs-build-2014\SymbolStore\Mitutoyo.MCOSMOS.BasicLibs_Release_NuGet
\end{lstlisting}

Und letztlich muss der Microsoft Server abgefragt werden, damit die Funktionen des Betriebssystems korrekt aufgelöst werden können.
\begin{lstlisting}[language=WinDbg]
0:000> .symfix+ d:\debug\symbols
\end{lstlisting}

Nach dem Ändern der möglichen Quellen für Symbole muss dem Debugger mitgeteilt werden, dass er seine Informationen aktualisiert.

\begin{lstlisting}
0:000> .reload /f
.*** WARNING: Unable to verify checksum for GEOPAK64.exe
.......*** WARNING: Unable to verify checksum for GeoWinBinToAsc64.dll
.....*** WARNING: Unable to verify checksum for Mafis_3DCmp64.dll
.......*** WARNING: Unable to verify checksum for MnGeoWnListCtrl64.dll
.*** WARNING: Unable to verify checksum for MnRecordPoints64.dll
.*** WARNING: Unable to verify checksum for UncertaintyCalculator64.dll
..

Press ctrl-c (cdb, kd, ntsd) or ctrl-break (windbg) to abort symbol loads that take too long.
Run !sym noisy before .reload to track down problems loading symbols.

[...]
Loading unloaded module list
..........................

************* Symbol Loading Error Summary **************
Module name            Error
WkWin64                The system cannot find the file specified
GeoWinBinToAsc64       The system cannot find the file specified
[...]

You can troubleshoot most symbol related issues by turning on symbol loading diagnostics (!sym noisy) and repeating the command that caused symbols to be loaded.
You should also verify that your symbol search path (.sympath) is correct.
\end{lstlisting}

Anhand der beiden bekannten und für die Fehleranalyse notwendigen Module Geopak und MnGeom3 kann überprüft werden, ob die Symbole korrekt geladen wurden.

\begin{lstlisting}
0:000> lm m geopak*
Browse full module list
start             end                 module name
00007ff7`fa2d0000 00007ff7`fc142000   GEOPAK64 C (private pdb symbols)  d:\temp\bug 32147\GEOPAK64.pdb

0:000> lm m mngeom364
Browse full module list
start             end                 module name
00007ff9`95c00000 00007ff9`95c2c000   MnGeom364 C (private pdb symbols)  d:\temp\bug 32147\MnGeom364.pdb
\end{lstlisting}

Für beide Module sind Symbole mit Informationen zu privaten Methoden etc. vorhanden.

\subsection{Exception Analyse}

Wie bereits bei den grundlegenden Prüfungen gesehen, handelt es sich beim Absturz um eine Access Violation, also eine Art NullPointerException. Nach dem Einstellen der Symbole verschwindet allerdings die Warnung.

\begin{lstlisting}
0:000> .exr -1
ExceptionAddress: 00007ff995c1d48e (MnGeom364!tg_ajc_lin_int+0x000000000000014e)
   ExceptionCode: c0000005 (Access violation)
  ExceptionFlags: 00000000
NumberParameters: 2
   Parameter[0]: 0000000000000000
   Parameter[1]: 0000000000000000
Attempt to read from address 0000000000000000
\end{lstlisting}

Der Callstack liefert nicht die richtigen Angaben

\begin{lstlisting}
0:000> k
 # Child-SP          RetAddr               Call Site
00 00000056`12337058 00000193`813b1bec     ntdll!NtGetContextThread+0x14
01 00000056`12337060 0000001f`00000002     0x00000193`813b1bec
02 00000056`12337068 0000003e`0000003e     0x0000001f`00000002
03 00000056`12337070 00009c67`02cb62cf     0x0000003e`0000003e
04 00000056`12337078 00009c67`02cb7d3f     0x00009c67`02cb62cf
05 00000056`12337080 00000000`00000000     0x00009c67`02cb7d3f
\end{lstlisting}

Dies bedeutet, dass der Kontext noch nicht auf die Exception gesetzt ist.

\begin{lstlisting}
0:000> .ecxr
rax=0000000000000000 rbx=000000561233b3f8 rcx=000000561233af18
rdx=ffffffffffffffe0 rsi=000000561233b490 rdi=000000561233b470
rip=00007ff995c1d48e rsp=000000561233afe0 rbp=000000561233b0e0
 r8=00000193a1a21c90  r9=0000000000000014 r10=000000000000003c
r11=000000561233afd0 r12=00000193a1a21c90 r13=000000561233b608
r14=0000000000000014 r15=0000000000000001
iopl=0         nv up ei pl nz na pe nc
cs=0033  ss=002b  ds=002b  es=002b  fs=0053  gs=002b             efl=00010202
MnGeom364!tg_ajc_lin_int+0x14e:
00007ff9`95c1d48e 0f1000          movups  xmm0,xmmword ptr [rax] ds:00000000`00000000=????????????????????????????????
\end{lstlisting}

Nach dem Setzen des Kontexts wird die Methode \verb|tg_ajc_lin_int| auf dem Stack erkannt.

\begin{lstlisting}
0:000> k Lb
  *** Stack trace for last set context - .thread/.cxr resets it
 # Child-SP          RetAddr               Call Site
00 00000056`1233afe0 00007ff9`95c1d8b8     MnGeom364!tg_ajc_lin_int+0x14e [d:\gitrepos\mitutoyo.mcosmos.basicslibs\source\mngeom3\tg_geom.c @ 955] 
01 00000056`1233b350 00007ff9`95c1fefb     MnGeom364!tg_ajc_lin_rl+0xd8 [d:\gitrepos\mitutoyo.mcosmos.basicslibs\source\mngeom3\tg_geom.c @ 993] 
02 00000056`1233b560 00007ff7`fae310df     MnGeom364!tg_inn_ajc_lin+0x2b [d:\gitrepos\mitutoyo.mcosmos.basicslibs\source\mngeom3\tg_geom.c @ 1026] 
03 00000056`1233b6b0 00007ff7`fae4a291     GEOPAK64!pel_line+0x77f [g:\git\mcosmos50\geopak\source\geopak\pelmadcp.cpp @ 5498] 
04 00000056`1233bf00 00007ff7`fae49687     GEOPAK64!pelm_comp_elem_intern+0xc01 [g:\git\mcosmos50\geopak\source\geopak\pelmadcp.cpp @ 8255] 
05 00000056`1233e730 00007ff7`fa7f5a88     GEOPAK64!pelm_comp_elem+0x77 [g:\git\mcosmos50\geopak\source\geopak\pelmadcp.cpp @ 8651] 
06 00000056`1233e790 00007ff7`fa920b77     GEOPAK64!fctctr_cpnt_end+0x748 [g:\git\mcosmos50\geopak\source\geopak\fctctrmngr.cpp @ 1376] 
07 00000056`1233efe0 00007ff7`fa91d26c     GEOPAK64!fctmngr_wrk_fct+0x1177 [g:\git\mcosmos50\geopak\source\geopak\fct_mngr.cpp @ 1314] 
08 00000056`1233f070 00007ff7`faaf8df3     GEOPAK64!fctmngr_wrk+0x68c [g:\git\mcosmos50\geopak\source\geopak\fct_mngr.cpp @ 1924] 
09 00000056`1233f120 00007ff7`fabd9301     GEOPAK64!CGeopakDoc::PartProgCmdWrk+0x13 [g:\git\mcosmos50\geopak\source\geopak\geopakdoc.cpp @ 3711] 
0a 00000056`1233f150 00007ff9`ab5287f9     GEOPAK64!CMainFrame::OnMsgNvUserEvent+0x281 [g:\git\mcosmos50\geopak\source\geopak\mainfrm.cpp @ 2215] 
\end{lstlisting}

Ausgehend von einer Nutzerinteraktion (\verb|OnMsgNvUserEvent|) werden mehrere Methoden aufgerufen, die dann zum Absturz führen.



\lstdefinelanguage{VS}{
    alsoletter={<, >, /, \#, _},
    keywords=[1]{\#define, \#include},
    keywordstyle=[1]\color{gray},
    keywords=[2]{<experimental/filesystem>},
    keywordstyle=[2]\color{lorange},
    keywords=[3]{_SILENCE_EXPERIMENTAL_FILESYSTEM_DEPRECATION_WARNING},
    keywordstyle=[3]\color{purple},
    sensitive=false, % keywords are not case-sensitive
    morecomment=[l]{}, % l is for line comment
    morecomment=[s]{/*}{*/}, % s is for start and end delimiter
    morestring=[b]" % strings are enclosed in double quotes
}%
\lstset{language=VS}

\section{Build Instructions}
\label{sec:buildinstructions}

\subsection{WinDbg Crash Analyse}
\label{subsec:build_windbg}

Benötigte Tools:
\begin{itemize}
    \item Windows Betriebssystem
    \item WinDbg Preview \cite{misc:windbgpreview}
    \item Crash Dumb des Bugs\footnotemark
    \item Symbol-Dateien des Codes -> GEOPAK64.pdb, MnGeom364.pdbS \footnotemark[\value{footnote}] \footnotetext{Erreichbar unter H:\textbackslash thomasW\textbackslash v\_DB}
    \item Zugriff auf Azure DevOps Server von Mitutoyo-CTL Germany GmbH
\end{itemize}

Um den Crash Dump nun zu analysieren, wird dieser mit WinDbg Preview über die Funktion \glqq{}Open dump file\grqq{} geöffnet.
Die darauf folgenden Schritte sind bereits in \ref{sec:windbg} beschrieben.


\subsection{Kompilieren von Mitutoyo.MCOSMOS.BasicLibs}
\label{subsec:build_code}

Benötigte Tools:
\begin{itemize}
    \item Windows Betriebssystem
    \item Visual Studio 2017 oder Visual Studio 2019 ohne Updates
    \item Zugriff auf den Azure DevOps Serever von Mitutoyo-CTL Germansy GmbH
    \item Git Extensions oder ein ähnliches Tool zum Klonen von Git Repositorys
\end{itemize}

Damit der Source Code kompiliert werden kann, muss dieser zuest mit Git Extensions (oder einem ähnlichen Tool) von
\url{http://tfs-ctlg.mitutoyo-ctl.de/Public/Projects/_git/Mitutoyo.MCOSMOS.BasicsLibs} geklont werden. Damit der Bug noch vorhanden ist, sollte der Source Code vom Stand des
Commits \verb|bb882af1| verwendet werden. \newline
Anschließend öffnet man die Projekt-Datei \textit{Mitutoyo.MCOSMOS.BasicsLibs\textbackslash Source\textbackslash MBasicsLibs.sln} mit Visual Studio.

\begin{figure}[htpb]
    \centering
    \includegraphics[width=0.85\textwidth]{x64dll_correction}
    \caption{Durchzuführende Korrektur}
    \label{img:x64dll_correction}
\end{figure}

Da \gls{masm} keine Leerzeichen im Pfadname verarbeiten kann, muss in den Eigenschaften der Datei
\textit{Mitutoyo.MCOSMOS.BasicsLibs\textbackslash Source\textbackslash CryptoPP\textbackslash x64dll.asm} unter \glqq{}Custom Build Tool\grqq{}
-> \glqq{}General\grqq{} -> \glqq{}Command Line\grqq{} folgender Eintrag eingegeben werden: \textit{ml64.exe /c /nologo /Fo"\$(IntDir)x64dll.obj" /Zi "\%(FullPath)"}
(vgl. \ref{img:x64dll_correction}).

Des Weiteren muss in der Datei \textit{IniFileHelper.h} folgender Code ergänzt werden:
\begin{lstlisting}
14    #define _SILENCE_EXPERIMENTAL_FILESYSTEM_DEPRECATION_WARNING
15    #include <experimental/filesystem>
\end{lstlisting}

Zuletzt muss die Operation \glqq{}Restore NuGet Packages\grqq{}, über einen Rechtsklick auf die Solution, ausgewählt und ausgeführt werden.

Der Code kann nun kompiliert werden.


\printbibliography          % Literaturverzeichnis
\listoftodos                % Todoverzeichnis

\end{document}