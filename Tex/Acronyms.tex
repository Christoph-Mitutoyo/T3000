%%%%%%%%%%%%% Abkürzungen %%%%%%%%%%%%%

\newacronym{ip}{IP}{Internet Protocol}

\newacronym{dbms}{DBMS}{Datenbankmanagementsystem}
\newacronym{sql}{SQL}{Structured Query Language}
\newacronym{oql}{OQL}{Object Query Language}
\newacronym{ui}{UI}{User Interface}
\newacronym{dhbw}{DHBW}{Duale Hochschule Baden-Württemberg}
\newacronym{os}{OS}{Operating System}
\newacronym[plural=SDKs,firstplural=Software Development Kits(SDK)]{sdk}{SDK}{Software Development Kit}
\newacronym{rss}{RSS}{Rich Site Summary, bzw. Really Simple Syndication}
\newacronym{json}{JSON}{JavaScript Object Notation}
\newacronym{ram}{RAM}{Random Access Memory}
\newacronym{apk}{APK}{Android Package}
\newacronym[plural=APIs,firstplural=Application Programming Interfaces (APIs)]{api}{API}{Application Programming Interface}
\newacronym{url}{URL}{Uniform Resource Locator}
\newacronym{uid}{UID}{User Identification}
\newacronym{dsgvo}{DSGVO}{Datenschutz-Grundverordnung}
\newacronym{xml}{XML}{Extensible Markup Language}

%%%%%%%%%%%% T3000 %%%%%%%%%%%%
\newacronym{ub}{UB}{Undefined Behavior}
\newacronym{masm}{MASM}{Microsoft Macro Assembler}
\newacronym{ir}{IR}{\gls{intermediaterepresentation}}
\newacronym{ubsan}{UBSan}{Undefined Behavior Sanitizer}

\newglossaryentry{intermediaterepresentation}
{
    name={Intermediate Representation},
    description={Datenstruktur/Code, welche in einem Compiler benutzt wird um Source Code darzustellen}
}
\newglossaryentry{callingconvention}
{
    name={Calling Conventions},
    plural={Calling Conventions},
    description={Methode, mit welcher ein Programm Daten an ein Unterprogramm weitergibt}
}
\newglossaryentry{compiler}
{
    name={Compiler},
    description={Programm, welches Code einer Programmiersprache in Maschinencode umwandelt},
}
%%%%%%%%%%%% Glossar-Einträge %%%%%%%%%
\newglossaryentry{iostream}
{
    name={I/O-Stream},
    plural={I/O-Streams},
    description={Eingangs-/Ausgangsstrom, wandelt Eingabe/Ausgabe in eine für das/den Programm/Menschen verständliche Sprache um},
}
\newglossaryentry{listactivity}
{
    name={ListActivity},
    plural={ListActivities},
    description={Android Aktivität, welche eine Liste von Objekten darstellt},
}
\newglossaryentry{hashmap}
{
    name={Hashmap},
    plural={Hashmaps},
    description={Speicher für Key-Value-Paare},
}
\newglossaryentry{container}
{
    name={Container},
    plural={Container},
    description={Hier: Teil der Liste, welches ein einzelnes Objekt/Element enthält},
}
\newglossaryentry{debugger}
{
    name={Debugger},
    plural={Debugger},
    description={Werkzeug zum Finden von Fehlern im Code},
}
\newglossaryentry{listener}
{
    name={Listener},
    plural={Listener},
    description={Auf eine bestimmte Aktion registrierte Klasse, welche eine Funktion enthält. Die Funktion wird aufgerufen, sobald die auf den Listener registrierte Aktion ausgeführt wird},
}
\newglossaryentry{rssfeed}
{
    name={RSS-Feed},
    plural={RSS-Feeds},
    description={Abrufbare Quelle von häufig aktualisierten Informationen, wie Blogs und Nachrichten. Meist eine Zusammenfassung des originalen Inhalts},
}
\newglossaryentry{rssreader}
{
    name={RSS-Reader},
    plural={RSS-Reader},
    description={Anwendung, welche Daten aus einem \gls{rssfeed} liest},
}
\newglossaryentry{cloud}
{
    name={Cloud},
    plural={Clouds},
    description={Über das Internet verfügbare IT-Infrastruktur, meist zu Speicherzwecken},
}
\newglossaryentry{popup}
{
    name={Pop-up},
    plural={Pop-ups},
    description={Element der grafischen Oberfläche, welches eingesetzt wird um Zusatzinhalte anzuzeigen (engl. \glqq{}to pop up\grqq{}, \glqq{}plötzlich auftauchen\grqq{})},
}
\newglossaryentry{userstory}
{
    name={User Story},
    plural={User Stories},
    description={Aus Nutzersicht beschriebene Anforderungen}
}
\newglossaryentry{abfrsprache}
{
    name={Abfragesprache},
    description={Datenbanksprache zur Informationssuche; je nach Datenbankmodell verschieden},
    plural={Abfragesprachen}
}

\newglossaryentry{nosql}
{
    name = {NoSQL},
    description = {Datenbankmodell, welches keinen relationalen (\gls{sql}) Ansatz verfolgt},
}

\newglossaryentry{smartwatch}
{
    name = {Smartwatch},
    description = {Elektronische Armbanduhr, welche mit einem Smartphone verbunden werden kann und wie dieses bedient wird.},
    plural = {Smartwatches}
}


\newglossaryentry{tposdat}
{
    name={ToolPos.dat Datei},
    description={Datei mit notwendigen Informationen für einen automatischen Tasterwechsel},
    first = {\glsentrydesc{tposdat} (\glsentrytext{tposdat})},
    plural = {ToolPos.dat Dateien},
    descriptionplural={Dateien mit notwendigen Informationen für einen automatischen Tasterwechsel},
    firstplural={\glsentrydescplural{tposdat} (\glsentryplural{tposdat})}
}
\newglossaryentry{d3Coord}{
    name={3D-Koordinate},
    plural={3D-Koordinaten},
    first={\glsentrydesc{d3Coord} (\glsentrytext{d3Coord})},
    firstplural={\glsentrydescplural{d3Coord} (\glsentryplural{d3Coord})},
    description={Kartesische Koordinate im dreidimensionalen Raum},
    descriptionplural={Kartesische Koordinaten im dreidimensionalen Raum}
}
\newglossaryentry{cmmsm}
{
    name={CMM Status Monitor},
    %first={\glsentrydesc{cmmsm} (\glsentrytext{cmmsm})},
    description={\gls{cmm} Status Monitor überwacht den aktuellen Betriebsstatus der Maschine und übermittelt diesen an den Nutzer}
}
\newglossaryentry{ws}
{
    name = {Werkstück},
    plural = {Werkstücke},
    first = {\glsentrydesc{ws} (\glsentrytext{ws})},
    firstplural={\glsentrydescplural{ws} (\glsentryplural{ws})},
    description = {Abgegrenztes Bauteil, welches in irgendeiner Form weiterverarbeitet wird},
    descriptionplural = {abgegrenzte Bauteile, welche in irgendeiner Form weiterverarbeitet werden}
}
\newglossaryentry{taster}
{
    name = {Taster},
    plural = {Taster},
    description = {Meist mit Rubinkugel ausgerüstetes Teil des \gls{kmg}, welches das Wekstück antastet},
}
\newglossaryentry{tbaum}
{
    name = {Tasterbaum},
    plural = {Tasterbäume},
    description = {Zusammensetzung mehrerer Taster zu einem System},
    descriptionplural = {Zusammensetzung mehrerer Taster zu einem System}
}
\newglossaryentry{mksystem}
{
    name = {Messkopfsystem},
    plural = {Messkopfsysteme},
    description = {gesamten Systems aus \gls{tbaum} und Messkopf},
    descriptionplural = {Gesamtes System aus \gls{tbaum} und Messkopf},
    first = {\glsentrydesc{mksystem} (\glsentrytext{mksystem})},
}
\newglossaryentry{mpg}
{
    name = {Messprogramm},
    plural = {Messprogramme},
    description = {Von der Software erstellte Instruktionen, welche am KMG ausgeführt werden; Messvorgänge am Werkstück}
}
\newglossaryentry{cleancode}
{
    name = {Clean-Code},
    plural = {Clean-Code},
    description = {Prinzipien und Praktiken zu mehr Qualität in der Softwareentwicklung}
}
\newglossaryentry{messvl}
{
    name = {Messvolumen},
    plural = {Messvolumen},
    description = {Bereich, in dem sich das Messkopfsystem bewegen kann um Werkstücke zu messen}
}
\newglossaryentry{atmega}
{
    name = {ATmega328},
    description = {Von Atmel entwickelter Mikroprozessor}
}
\newglossaryentry{opensource}
{
    name = {Open-Source-Software},
    description = {Software mit frei zugänglichem Quelltext},
    first = {\glsentrydesc{opensource} (\glsentrytext{opensource})}
}
\newglossaryentry{bruecke}
{
    name = {Brücke},
    plural = {Brücken},
    description = {Brückenförmiger Teil eines \gls{kmg}, welcher in y-Richtung verfahren wird}
}
\newglossaryentry{pinole}
{
    name = {Pinole},
    plural = {Pinolen},
    description = {Beweglicher Teil eines \gls{kmg}, welcher den \gls{tbaum} trägt}
}
\newglossaryentry{rrdb}
{
    name = {Round-Robin-Datenbank},
    plural = {Round-Robin-Datenbanken},
    description = {Datenbank mit vordefinierter Speicherzeit (Dateien, die diese überschreiten, werden mit neuen Dateien überschrieben)}
}
\newglossaryentry{mctl}
{
    name = {Mitutoyo CTL GmbH},
    description = {Mitutoyo Computer Technology Laboratory ist ein Tochterunternehmen der Mitutoyo Europe GmbH und ist auf die Entwicklung der Software für KMGs spezialisiert}
}
\newglossaryentry{eeprom}
{
    name = {EEPROM},
    description = {Electrically Erasable Programmable Read-Only Memory (EEPROM); elektronischer Speicherbaustein, dessen gespeicherte Daten elektrisch gelöscht werden können}
}
%Hier Label von der längsten Abkürzung eintragen
\glsfindwidesttoplevelname