\chapter{Anhang}
\label{ch:anhang}

\section{Personas}
\label{sec:personas}

Eine Persona ist eine fiktionale Person. Das erstellen einer Persona kann helfen weitere Anforderungen an ein Projekt zu identifizieren, indem Personas mit unterschiedlichen Ausgangssituationen und Interessen kreiert werden.

\pagebreak
\subsection{Daniel Dacher}
\label{subsec:danieldacher}

\begin{figure}[htpb]
    \centering
    \includegraphics[width=0.5\textwidth]{DanielDacher}
    \caption{Generiertes Porträt von Daniel Dacher \cite{misc:danieldacher}}
    \label{img:danieldacher}
\end{figure}
\textbf{Name:}
\par
\begingroup
\leftskip=30pt
\noindent
Daniel Dacher \newline
Anfangsbuchstabe D wie Datenschutzbeauftragter.
\par
\endgroup
\textbf{Relevanz:}
\par
\begingroup
\leftskip=30pt
\noindent
Ein Datenschutzbeauftragter wurde in der ersten Version der Stakeholderanalyse nicht berücksichtigt.
\par
\endgroup
\textbf{Wohnort:}
\par
\begingroup
\leftskip=30pt
\noindent
Suttgart
\par
\endgroup
\textbf{Firma:}
\par
\begingroup
\leftskip=30pt
\noindent
Wir-schützen-Daten GmbH

Die Firma untersucht neue Softwarelösungen auf deren Verträglichkeit mit der DSGVO. Dies geschieht entweder Stichprobenweise oder auf Kundenanfrage.
\par
\endgroup
\textbf{Arbeitsort:}
\par
\begingroup
\leftskip=30pt
\noindent
Stuttgart
\par
\endgroup
\textbf{Alter:}
\par
\begingroup
\leftskip=30pt
\noindent
37 Jahre
\par
\endgroup
\textbf{Familienstand:}
\par
\begingroup
\leftskip=30pt
\noindent
Verlobt
\par
\endgroup
\textbf{Kenntnisse unnd Fähigkeiten:}
\par
\begingroup
\leftskip=30pt
\noindent
Ausbildung zum Datenschutzbeauftragten \newline
DSGVO Datenschutzbeauftragter Zertifizierung
\par
\endgroup

Aus der Persona \glqq{}Daniel Dacher\grqq{} geht hervor, dass auf den Umgang mit personenbezogenen Daten geachtet werden muss. \newline
Dies ist insbesondere relevant, da die Anforderung A-8 das Erstellen von persönlichen Profilen fordert.


\pagebreak
\subsection{Susanne Schmid}
\label{subsec:susanneschmid}

\begin{figure}[htpb]
    \centering
    \includegraphics[width=0.5\textwidth]{SusanneSchmid}
    \caption{Generiertes Porträt von Susanne Schmid \cite{misc:susanneschmid}}
    \label{img:susanneschmid}
\end{figure}

\textbf{Name:}
\par
\begingroup
\leftskip=30pt
\noindent
Susanne Schmid \newline
Standardbenutzerin von Strike Up mit dem Anfangsbuchstaben S.
\par
\endgroup
\textbf{Relevanz:}
\par
\begingroup
\leftskip=30pt
\noindent
Stellt den voraussichtlichen Durchschnittsbenutzer von Strike Up dar.
\par
\endgroup
\textbf{Wohnort:}
\par
\begingroup
\leftskip=30pt
\noindent
Ehrenfeld, Köln \newline
Ehrenfeld ist ein Stadteil von Köln mit ungefähr 37.500 Einwohnern.
\par
\endgroup
\textbf{Firma:}
\par
\begingroup
\leftskip=30pt
\noindent
Lebensmittelladen \newline
Lebensmittelladen ist eine kleinere Version von ähnlichen Firmen, wie Aldi, Netto, Lidl und weiteren.
\par
\endgroup
\textbf{Arbeitsort:}
\par
\begingroup
\leftskip=30pt
\noindent
Chorweiler, Köln \newline
Der Weg zur Arbeit mit öffentlichen Verkehrsmitteln dauert ungefähr 30 Minuten.
\par
\endgroup
\textbf{Jobtitel/Rollen:}
\par
\begingroup
\leftskip=30pt
\noindent
Susanne ist Assistant Managerin von Lebensmittelladen. Sie kümmert sich um den Vertrieb, geht gelegentlich aber auch an die Kasse.
\par
\endgroup
\textbf{Arbeitszeit:}
\par
\begingroup
\leftskip=30pt
\noindent
40h-Woche typisch von 7:00-16:00
\par
\endgroup
\textbf{Alter:}
\par
\begingroup
\leftskip=30pt
\noindent
35 Jahre
\par
\endgroup
\textbf{Familienstand:}
\par
\begingroup
\leftskip=30pt
\noindent
Verheiratet, zwei Kinder
\par
\endgroup
\textbf{Kenntnisse unnd Fähigkeiten:}
\par
\begingroup
\leftskip=30pt
\noindent
Ausbildung bei Lebensmittelladen
\par
\endgroup
\textbf{Persönliche Eigenschaften:}
\par
\begingroup
\leftskip=30pt
\noindent
leicht reizbar/ungeduldig, hauptsächlich machen sich diese Eigenschaften wegen der frühen Arbeitszeit bemerkbar. Dies könnte zu allgemein schlechteren Reaktionen auf Vorschläge von Strike Up und das mehrfache Klicken von Buttons führen. \newline
offen für Neues, Susanne ist interessiert daran neue Menschen zu treffen und neue Dinge auszuprobieren \newline
schüchtern/zurückhaltend, ihre schüchterheit hält Susanne oft davon ab ein Gespräch zu beginnen
\par
\endgroup

\pagebreak
\subsection{Torsten Tacheles}
\label{subsec:torstentacheles}

\begin{figure}[htpb]
    \centering
    \includegraphics[width=0.5\textwidth]{TorstenTacheles}
    \caption{Generiertes Porträt von Torsten Tacheles \cite{misc:torstentacheles}}
    \label{img:torstentacheles}
\end{figure}

\textbf{Name:}
\par
\begingroup
\leftskip=30pt
\noindent
Torsten Tacheles \newline
Anfangsbuchstabe T wie Trainer.
\par
\endgroup
\textbf{Relevanz:}
\par
\begingroup
\leftskip=30pt
\noindent
Kommunikationstrainer der Strike Up benutzt um seinen Kenntnisstand zu erweitern und neue Ideen anzureichern. \newline
Kann Strike Up an Kunden weiterempfehlen. \newline
Strike Up war ursprünglich nur zur Gesprächsführung gedacht.
\par
\endgroup
\textbf{Wohnort:}
\par
\begingroup
\leftskip=30pt
\noindent
Berlin
\par
\endgroup
\textbf{Firma:}
\par
\begingroup
\leftskip=30pt
\noindent
Kommunikation Lernen 1on1 \newline
Torsten hat die Firma gemeinsam mit einem Freund gegründet. Es gibt keine weiteren Mitarbeiter. \newline
Die Firma ist darauf spezialisiert Menschen Kommunikationstechniken und allgemeine Rhetorik beizubringen.
\par
\endgroup
\textbf{Arbeitsort:}
\par
\begingroup
\leftskip=30pt
\noindent
Berlin
\par
\endgroup
\textbf{Jobtitel/Rollen:}
\par
\begingroup
\leftskip=30pt
\noindent
Torsten ist der Chef der Firma. Er kümmert sich neben Trainingskursen auch um alle weiteren Angelegenheiten der Firma.
\par
\endgroup
\textbf{Arbeitszeit:}
\par
\begingroup
\leftskip=30pt
\noindent
45h-Woche typisch von 7:00-17:00
\par
\endgroup
\textbf{Alter:}
\par
\begingroup
\leftskip=30pt
\noindent
57 Jahre
\par
\endgroup
\textbf{Familienstand:}
\par
\begingroup
\leftskip=30pt
\noindent
Ledig
\par
\endgroup
\textbf{Kenntnisse unnd Fähigkeiten:}
\par
\begingroup
\leftskip=30pt
\noindent
Studium der Psychologie \newline
Weiterbildung zum Kommunikationstrainer \newline
Online-Seminare um die Kenntnisse zum Leiten einer Firma zu erlangen.
\par
\endgroup
\textbf{Persönliche Eigenschaften:}
\par
\begingroup
\leftskip=30pt
\noindent
gesprächsfreudig, Torstens Job ist es Menschen Kommunikationstechniken beizubringen, es ist für ihn gewissermaßen Pflicht gerne zu reden. \newline
zynisch, Torstens Zynismus hat ihn manchmal bereits Kunden gekostet
\par
\endgroup
