\lstdefinelanguage{WinDbg}{
   alsoletter={., |, +, :, 0, >, !, ?},
   keywords=[1]{.time, .exr, .sympath, .sympath+, .symfix, .symfix+, .symopt, .symopt+, |, ||, .reload, lm, !sym, k, .excr, .ecxr, .thread, .cxr, .frame, dv, dD, dx, dq, ?},
   keywordstyle=[1]\color{dkgreen},
   keywords=[2]{0:000>},
   keywordstyle=[2]\color{gray},
   sensitive=false, % keywords are not case-sensitive
   morecomment=[l]{}, % l is for line comment
   morecomment=[s]{/*}{*/}, % s is for start and end delimiter
   morestring=[b]" % strings are enclosed in double quotes
} % 
\lstset{language=WinDbg}

\section{WinDbg Crash Analyse}
\label{sec:windbg}
Vom betroffenen Absturz, der womöglich auf Undefined Behavior zurückzuführen ist, wurde ein Crash Dump erstellt.
Dieser Anhang beschreibt anhand einer aufgezeichneten Logdatei, welche Befehle genutzt wurden, um den Crash Dump zu analysieren.
Bei der analysierten Datei handelt es sich um einen Crash Dump mit vollständig enthaltenem Speicherinhalt. Die Datei ist ca. 1 GB groß.
\subsection{Grundlegende Prüfungen}
\begin{lstlisting}
0:000> ||
.  0 Full memory user mini dump: D:\MINIDUMP-20201217-150359.DMP
\end{lstlisting}
Dem Dateinamen nach wurde der Crash Dump am 17.12.2020 um 15:03:59 Uhr UTC geschrieben. Die Zeitangabe innerhalb des Crash Dumps bestätigt dies.

\begin{lstlisting}
0:000> .time
Debug session time: Thu Dec 17 16:04:00.000 2020 (UTC + 1:00)
System Uptime: 0 days 7:25:08.968
Process Uptime: 0 days 0:01:31.000
  Kernel time: 0 days 0:00:32.000
  User time: 0 days 0:00:48.000
\end{lstlisting}
Der Bug wurde am 27.11.2020 berichtet und am 11.12.2020 erstmals bestätigt. Beim vorliegenden Crash Dump handelt es sich also um die Reproduktion des Fehlers zu einem späteren Zeitpunkt.

\begin{lstlisting}
0:000> |
.  0	id: 2b7c	examine	name: C:\MCOSMOSx64\EXE\GEOPAK64.exe
\end{lstlisting}
Das ausgeführte Programm stimmt mit dem des Fehlerberichts überein.

\begin{lstlisting}[language=WinDbg]
0:000> .exr -1
*** WARNING: Unable to verify checksum for MnGeom364.dll
ExceptionAddress: 00007ff995c1d48e (MnGeom364!tg_ajc_lin_int+0x000000000000014e)
   ExceptionCode: c0000005 (Access violation)
  ExceptionFlags: 00000000
NumberParameters: 2
   Parameter[0]: 0000000000000000
   Parameter[1]: 0000000000000000
Attempt to read from address 0000000000000000
\end{lstlisting}
Der Fehlercode und das fehlerhafte Modul stimmen ebenfalls mit dem des Fehlerberichts überein. Es kann also eine detailliertere Analyse durchgeführt werden.

\subsection{Symbole}
Für eine weitere Analyse ist das Vorhandensein von Symbolen erforderlich. Manche Symbole wurden zusammen mit dem Crash Dump abgelegt.
Diese Symbole sollten zuerst berücksichtigt werden und müssen zuerst in den Symbolpath aufgenommen werden.
\begin{lstlisting}[language=WinDbg]
0:000> .sympath "D:\temp\Bug 32147"
Symbol search path is: D:\temp\Bug 32147
Expanded Symbol search path is: d:\temp\bug 32147

************* Path validation summary **************
Response                         Time (ms)     Location
OK                                             D:\temp\Bug 32147
\end{lstlisting}

Leider stimmen bei den Symbolen die Zeitstempel nicht überein, so dass die Symbole nicht geladen werden können.
Die Prüfung des Zeitstempels kann jedoch ausgeschaltet werden.
\begin{lstlisting}[language=WinDbg]
0:000> .symopt+ 0x40
Symbol options are 0x30377:
  0x00000001 - SYMOPT_CASE_INSENSITIVE
  0x00000002 - SYMOPT_UNDNAME
  0x00000004 - SYMOPT_DEFERRED_LOADS
  0x00000010 - SYMOPT_LOAD_LINES
  0x00000020 - SYMOPT_OMAP_FIND_NEAREST
  0x00000040 - SYMOPT_LOAD_ANYTHING
  0x00000100 - SYMOPT_NO_UNQUALIFIED_LOADS
  0x00000200 - SYMOPT_FAIL_CRITICAL_ERRORS
  0x00010000 - SYMOPT_AUTO_PUBLICS
  0x00020000 - SYMOPT_NO_IMAGE_SEARCH
\end{lstlisting}

Weitere Symbole können vom Azure DevOps Server geladen werden
\begin{lstlisting}[language=WinDbg]
0:000> .sympath+ srv*d:\debug\symbols*\\tfs-build-2014\SymbolStore\Mitutoyo.MCOSMOS.BasicLibs_Release_NuGet
Symbol search path is: D:\temp\Bug 32147;srv*d:\debug\symbols*\\tfs-build-2014\SymbolStore\Mitutoyo.MCOSMOS.BasicLibs_Release_NuGet
Expanded Symbol search path is: d:\temp\bug 32147;srv*d:\debug\symbols*\\tfs-build-2014\symbolstore\mitutoyo.mcosmos.basiclibs_release_nuget

************* Path validation summary **************
Response                         Time (ms)     Location
OK                                             D:\temp\Bug 32147
Deferred                                       srv*d:\debug\symbols*\\tfs-build-2014\SymbolStore\Mitutoyo.MCOSMOS.BasicLibs_Release_NuGet
\end{lstlisting}

Und letztlich muss der Microsoft Server abgefragt werden, damit die Funktionen des Betriebssystems korrekt aufgelöst werden können.
\begin{lstlisting}[language=WinDbg]
0:000> .symfix+ d:\debug\symbols
\end{lstlisting}

Nach dem Ändern der möglichen Quellen für Symbole muss dem Debugger mitgeteilt werden, dass er seine Informationen aktualisiert.

\begin{lstlisting}
0:000> .reload /f
.*** WARNING: Unable to verify checksum for GEOPAK64.exe
.......*** WARNING: Unable to verify checksum for GeoWinBinToAsc64.dll
.....*** WARNING: Unable to verify checksum for Mafis_3DCmp64.dll
.......*** WARNING: Unable to verify checksum for MnGeoWnListCtrl64.dll
.*** WARNING: Unable to verify checksum for MnRecordPoints64.dll
.*** WARNING: Unable to verify checksum for UncertaintyCalculator64.dll
..

Press ctrl-c (cdb, kd, ntsd) or ctrl-break (windbg) to abort symbol loads that take too long.
Run !sym noisy before .reload to track down problems loading symbols.

[...]
Loading unloaded module list
..........................

************* Symbol Loading Error Summary **************
Module name            Error
WkWin64                The system cannot find the file specified
GeoWinBinToAsc64       The system cannot find the file specified
[...]

You can troubleshoot most symbol related issues by turning on symbol loading diagnostics (!sym noisy) and repeating the command that caused symbols to be loaded.
You should also verify that your symbol search path (.sympath) is correct.
\end{lstlisting}

Anhand der beiden bekannten und für die Fehleranalyse notwendigen Module Geopak und MnGeom3 kann überprüft werden, ob die Symbole korrekt geladen wurden.

\begin{lstlisting}
0:000> lm m geopak*
Browse full module list
start             end                 module name
00007ff7`fa2d0000 00007ff7`fc142000   GEOPAK64 C (private pdb symbols)  d:\temp\bug 32147\GEOPAK64.pdb

0:000> lm m mngeom364
Browse full module list
start             end                 module name
00007ff9`95c00000 00007ff9`95c2c000   MnGeom364 C (private pdb symbols)  d:\temp\bug 32147\MnGeom364.pdb
\end{lstlisting}

Für beide Module sind Symbole mit Informationen zu privaten Methoden etc. vorhanden.

\subsection{Exception Analyse}

Wie bereits bei den grundlegenden Prüfungen gesehen, handelt es sich beim Absturz um eine Access Violation, also eine Art NullPointerException. Nach dem Einstellen der Symbole verschwindet allerdings die Warnung.

\begin{lstlisting}
0:000> .exr -1
ExceptionAddress: 00007ff995c1d48e (MnGeom364!tg_ajc_lin_int+0x000000000000014e)
   ExceptionCode: c0000005 (Access violation)
  ExceptionFlags: 00000000
NumberParameters: 2
   Parameter[0]: 0000000000000000
   Parameter[1]: 0000000000000000
Attempt to read from address 0000000000000000
\end{lstlisting}

Der Callstack liefert nicht die richtigen Angaben

\begin{lstlisting}
0:000> k
 # Child-SP          RetAddr               Call Site
00 00000056`12337058 00000193`813b1bec     ntdll!NtGetContextThread+0x14
01 00000056`12337060 0000001f`00000002     0x00000193`813b1bec
02 00000056`12337068 0000003e`0000003e     0x0000001f`00000002
03 00000056`12337070 00009c67`02cb62cf     0x0000003e`0000003e
04 00000056`12337078 00009c67`02cb7d3f     0x00009c67`02cb62cf
05 00000056`12337080 00000000`00000000     0x00009c67`02cb7d3f
\end{lstlisting}

Dies bedeutet, dass der Kontext noch nicht auf die Exception gesetzt ist.

\begin{lstlisting}
0:000> .ecxr
rax=0000000000000000 rbx=000000561233b3f8 rcx=000000561233af18
rdx=ffffffffffffffe0 rsi=000000561233b490 rdi=000000561233b470
rip=00007ff995c1d48e rsp=000000561233afe0 rbp=000000561233b0e0
 r8=00000193a1a21c90  r9=0000000000000014 r10=000000000000003c
r11=000000561233afd0 r12=00000193a1a21c90 r13=000000561233b608
r14=0000000000000014 r15=0000000000000001
iopl=0         nv up ei pl nz na pe nc
cs=0033  ss=002b  ds=002b  es=002b  fs=0053  gs=002b             efl=00010202
MnGeom364!tg_ajc_lin_int+0x14e:
00007ff9`95c1d48e 0f1000          movups  xmm0,xmmword ptr [rax] ds:00000000`00000000=????????????????????????????????
\end{lstlisting}

Nach dem Setzen des Kontexts wird die Methode \verb|tg_ajc_lin_int| auf dem Stack erkannt.

\begin{lstlisting}
0:000> k Lb
  *** Stack trace for last set context - .thread/.cxr resets it
 # Child-SP          RetAddr               Call Site
00 00000056`1233afe0 00007ff9`95c1d8b8     MnGeom364!tg_ajc_lin_int+0x14e [d:\gitrepos\mitutoyo.mcosmos.basicslibs\source\mngeom3\tg_geom.c @ 955] 
01 00000056`1233b350 00007ff9`95c1fefb     MnGeom364!tg_ajc_lin_rl+0xd8 [d:\gitrepos\mitutoyo.mcosmos.basicslibs\source\mngeom3\tg_geom.c @ 993] 
02 00000056`1233b560 00007ff7`fae310df     MnGeom364!tg_inn_ajc_lin+0x2b [d:\gitrepos\mitutoyo.mcosmos.basicslibs\source\mngeom3\tg_geom.c @ 1026] 
03 00000056`1233b6b0 00007ff7`fae4a291     GEOPAK64!pel_line+0x77f [g:\git\mcosmos50\geopak\source\geopak\pelmadcp.cpp @ 5498] 
04 00000056`1233bf00 00007ff7`fae49687     GEOPAK64!pelm_comp_elem_intern+0xc01 [g:\git\mcosmos50\geopak\source\geopak\pelmadcp.cpp @ 8255] 
05 00000056`1233e730 00007ff7`fa7f5a88     GEOPAK64!pelm_comp_elem+0x77 [g:\git\mcosmos50\geopak\source\geopak\pelmadcp.cpp @ 8651] 
06 00000056`1233e790 00007ff7`fa920b77     GEOPAK64!fctctr_cpnt_end+0x748 [g:\git\mcosmos50\geopak\source\geopak\fctctrmngr.cpp @ 1376] 
07 00000056`1233efe0 00007ff7`fa91d26c     GEOPAK64!fctmngr_wrk_fct+0x1177 [g:\git\mcosmos50\geopak\source\geopak\fct_mngr.cpp @ 1314] 
08 00000056`1233f070 00007ff7`faaf8df3     GEOPAK64!fctmngr_wrk+0x68c [g:\git\mcosmos50\geopak\source\geopak\fct_mngr.cpp @ 1924] 
09 00000056`1233f120 00007ff7`fabd9301     GEOPAK64!CGeopakDoc::PartProgCmdWrk+0x13 [g:\git\mcosmos50\geopak\source\geopak\geopakdoc.cpp @ 3711] 
0a 00000056`1233f150 00007ff9`ab5287f9     GEOPAK64!CMainFrame::OnMsgNvUserEvent+0x281 [g:\git\mcosmos50\geopak\source\geopak\mainfrm.cpp @ 2215] 
\end{lstlisting}

Ausgehend von einer Nutzerinteraktion (\verb|OnMsgNvUserEvent|) werden mehrere Methoden aufgerufen, die dann zum Absturz führen.

\subsection{Extraktion von Daten für einen Unit Test}

Der Code der betroffenen Stelle aus \verb|TG_GEOM.C|, Commit \verb|f5d4dbb8|.

\begin{lstlisting}[language=C, numbers=left, firstnumber=937]
INTERN  MinMaxDist_T tg_ajc_lin_int(UInt4    ulNoofPnts,
                                    VecR8 LP aPoints,
                                    const VecR8 LP pMatDir,
                                    VecR8 LP pLoc,
                                    VecR8 LP pDir,
                                    Bool     DirLam,
                                    Bool     bFitLin)
/*************************************************************************
*************************************************************************/
{Real8        Tmp, Dist, MxD = 0.0;
 Bool         bCond;
 VecR8        Extremum, SupPnt=NVecR8, MatDir = *pMatDir, Dir = *pDir, Loc = *pLoc, LocBord;
 MinMaxDist_T Local = {0.0, 0.0, 0.0, NULL, NULL}, LocalBord = {0.0, 0.0, 0.0, NULL, NULL};
 UInt4    uMaxRep = 1;

if (ulNoofPnts<3) return Local;
pln_para_pnt (&MatDir, &Dist, &Loc);
Local = min_max_dist_pln (&MatDir, Dist, ulNoofPnts, aPoints);
Extremum = (bFitLin) ? *Local.MinPnt : *Local.MaxPnt;
if(bord_lin (&Loc, &Dir, &MxD, &Extremum, &SupPnt, ulNoofPnts, aPoints, DirLam))
   {
   do {
      LocBord = Loc;
      Extremum = SupPnt; uMaxRep++;
      MatDir = vec_norm (vec_vec_prod(Dir, vec_vec_prod(MatDir, Dir)), &Tmp);
      pln_para_pnt (&MatDir, &Dist, pLoc);
      LocalBord = min_max_dist_pln (&MatDir, Dist, ulNoofPnts, aPoints);
      bCond = bord_lin (&Loc, &Dir, &MxD, &Extremum, &SupPnt, ulNoofPnts, aPoints, DirLam);
      if(bCond && (LocalBord.MinMax < Local.MinMax))
       { *pDir = Dir; *pLoc = Loc; Local = LocalBord;}
      }
   while(
         bCond && 
         (uMaxRep < ulNoofPnts) && 
         (bg_dist_pntpnt(LocBord,Loc) > TG_EPS_PLN)
        );
   }
return Local;
}
\end{lstlisting}

Da der betroffene Code vor der Umstellung auf 64 Bit 12 Jahre lang nicht verändert wurde, wurde angenommen, dass nicht die Methode selbst, sondern die übergebenen Parameter für den Absturz verantwortlich sind.
Diese Daten können extrahiert werden.

\begin{lstlisting}
0:000> .frame 0
00 00000056`1233afe0 00007ff9`95c1d8b8     MnGeom364!tg_ajc_lin_int+0x14e [d:\gitrepos\mitutoyo.mcosmos.basicslibs\source\mngeom3\tg_geom.c @ 955] 
\end{lstlisting}

Der Befehl \verb|.frame| wechselt zu einem bestimmten Stackframe, hier dem obersten.

\begin{lstlisting}
0:000> dv
     ulNoofPnts = 0x14
        aPoints = 0x00000193`a1a21c90
        pMatDir = <value unavailable>
           pLoc = 0x00000056`1233b470
           pDir = 0x00000056`1233b490
         DirLam = True (0n1)
        bFitLin = True (0n1)
       Extremum = <value unavailable>
            Loc = struct VecR8
      LocalBord = <value unavailable>
           Dist = <value unavailable>
          bCond = <value unavailable>
         MatDir = struct VecR8
            MxD = 0
         SupPnt = struct VecR8
            Dir = struct VecR8
            Tmp = 0
        uMaxRep = 1
        LocBord = <value unavailable>
\end{lstlisting}

Aus dieser Ausgabe lassen sich bereits \verb|ulNoofPoints|, \verb|DirLam| und \verb|bFitLin| entnehmen.
Die Variable \verb|bFitLin| ist \verb|true|, d.h. in Zeile 955 wird \verb|*Local.MinPnt| dereferenziert. Dieser Wert wurde in Zeile 949 mit \verb|NULL| initialisiert und seither offenbar nicht geändert.

\begin{lstlisting}
0:000> dD /c 3 0x193a1a21c90 L0n60
00000193`a1a21c90                     -10                0.03674                      0
00000193`a1a21ca8                      -9                 0.0205                      0
00000193`a1a21cc0                      -8                0.01371                      0
00000193`a1a21cd8                      -7                  0.085                      0
00000193`a1a21cf0                      -6                 0.0286                      0
00000193`a1a21d08                      -5                0.02998                      0
00000193`a1a21d20                      -4               -0.05377                      0
00000193`a1a21d38                      -3                0.03051                      0
00000193`a1a21d50                      -2                0.06661                      0
00000193`a1a21d68                      -1               -0.09365                      0
00000193`a1a21d80                    -9.5               -0.09092                      0
00000193`a1a21d98                    -8.5                0.03863                      0
00000193`a1a21db0                    -7.5                0.04022                      0
00000193`a1a21dc8                    -6.5               -0.06106                      0
00000193`a1a21de0                    -5.5                0.05629                      0
00000193`a1a21df8                    -4.5               -0.00172                      0
00000193`a1a21e10                    -3.5                0.08086                      0
00000193`a1a21e28                    -2.5               -0.04595                      0
00000193`a1a21e40                    -1.5                -0.0227                      0
00000193`a1a21e58                    -0.5                0.01736                      0
\end{lstlisting}

Die übergebenen Punkte (\verb|aPoints|) sind nicht auffällig.

\begin{lstlisting}
0:000> dx -r1 (*((MnGeom364!VecR8 *)0x561233b250))
(*((MnGeom364!VecR8 *)0x561233b250))                 [Type: VecR8]
    [+0x000] x                : -0.002335 [Type: double]
    [+0x008] y                : -0.999997 [Type: double]
    [+0x010] z                : -0.000000 [Type: double]
\end{lstlisting}

Der Wert von \verb|pMatDir| ist nicht auffällig.

\begin{lstlisting}
0:000> dx -r1 ((MnGeom364!VecR8 *)0x561233b470)
((MnGeom364!VecR8 *)0x561233b470)                 : 0x561233b470 [Type: VecR8 *]
    [+0x000] x                : 0.000000 [Type: double]
    [+0x008] y                : -1.#QNAN0 [Type: double]
    [+0x010] z                : 4.750082 [Type: double]

0:000> dx -r1 (*((MnGeom364!VecR8 *)0x561233b040))
(*((MnGeom364!VecR8 *)0x561233b040))                 [Type: VecR8]
    [+0x000] x                : 0.000000 [Type: double]
    [+0x008] y                : -1.#QNAN0 [Type: double]
    [+0x010] z                : 4.750082 [Type: double]
\end{lstlisting}

Der Werte von \verb|pLoc| und dessen lokale Kopie \verb|Loc| enthalten allerdings den Sonderwert NaN (not a number).

\begin{lstlisting}
0:000> dx -r1 ((MnGeom364!VecR8 *)0x561233b490)
((MnGeom364!VecR8 *)0x561233b490)                 : 0x561233b490 [Type: VecR8 *]
    [+0x000] x                : 0.999997 [Type: double]
    [+0x008] y                : -0.002335 [Type: double]
    [+0x010] z                : 0.000000 [Type: double]
\end{lstlisting}

Der Wert für \verb|pDir| ist nicht auffällig.

\subsection{Mögliche Herkunft von NaN}

Der Speicherinhalt an Stelle \verb|0x561233b470| enthält den NaN Wert für die Eigenschaft \verb|y|.

\begin{lstlisting}
0:000> dx -r1 ((MnGeom364!VecR8 *)0x561233b470)
((MnGeom364!VecR8 *)0x561233b470)                 : 0x561233b470 [Type: VecR8 *]
    [+0x000] x                : 0.000000 [Type: double]
    [+0x008] y                : -1.#QNAN0 [Type: double]
    [+0x010] z                : 4.750082 [Type: double]
\end{lstlisting}

Im Fall einer nichtinitalisierten Variable könnte die Gleitkommazahl auch eine 64 Bit Ganzzahl gewesen sein.

\begin{lstlisting}
0:000> dq /c 3 0x561233b470 L3
00000056`1233b470  00000000`00000000 ffffffff`ffffffc8 40130015`997ff316

0:000> ? ffffffff`ffffffc8
Evaluate expression: -56 = ffffffff`ffffffc8
\end{lstlisting}

Falls das zutrifft, wäre ihr Wert -56 gewesen.