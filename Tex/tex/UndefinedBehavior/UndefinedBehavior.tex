\chapter{Undefined Behavior}
\label{ch:ub}

\section{Was ist Undefined Behavior?}
\label{sec:ub_was}

Der C++ Standard definiert \gls{ub} wie folgt: \newline
\glqq{}Permissible undefined behavior ranges from ignoring the situation completely with unpredictable results, to behaving during translation or program execution in a
documented manner characteristic of the environment [...]\grqq{}.\cite[S.8]{book:cpp-standard} \newline
Das Auftreten von \gls{ub} führt also entweder zu zufälligem Verhalten des Codes oder zu einem umgebungsspezifischem Verhalten.

Eben dieses zufällige Verhalten führte zu dem in \ref{sec:windbg} analysierten Bug. Ein Problem an zufälligem Verhalten ist, dass das Verhalten nicht zwingend zu einem
Absturz des Programms führt, sondern nur einen zufälligen Wert liefert. Dieser Wert kann nun Messergebnisse verfälschen und somit darauf schließen lassen, dass die gesamte
Software fehlerhaft ist.

\section{Ursachen von Undefined Behavior}
\label{sec:ub_ursachen}

Der C++ Standard\cite{book:cpp-standard} beschreibt über 500 verschieden Ursachen für \gls{ub}. Eine genauere Beschreibung aller Ursachen wird aufgrund der Anzahl nicht durchgeführt.

\textbf{Undefinierte Variablen:} Entgegen erster Annahmen, dass undefinierte Variablen zwingend zu \gls{ub} führen, ist dies seit 2014 nicht mehr der Fall, da sich die
\glspl{callingconvention} geändert haben. Der Wert einer undefinierten Variable ist nun als \glqq{}indeterminate\grqq{} festgelegt\cite[S.63]{book:cpp-standard}.
\gls{ub} tritt erst dann auf, wenn die undefinierte Variable ausgewertet wird (vgl. \ref{subsec:indeterminate}). Hierzu bestehen allerdings Ausnahmen, wenn \verb|unsigned|
Variablen verwendet werden. \cite[S.63]{book:cpp-standard}

\textbf{Nullpointer Dereferenzierung:} Der Versuch den Wert der Adresse eines Nullpointers zu lesen, endet in \gls{ub}. \cite[S.188]{book:cpp-standard} (vgl. \ref{subsec:nullpointer})

Einige weitere Ursachen von \gls{ub} sind:
\begin{itemize}
    \item Signed overflow \cite[S.74]{book:cpp-standard}
    \item Modifizieren eines \verb|const| Objekts \cite[S.174]{book:cpp-standard}
    \item Data races \cite[S.1376]{book:cpp-standard}
    \item Invalid pointer value \cite[S.67]{book:cpp-standard}
    \item Zugriff außerhalb der Lebensdauer \cite[S.35]{book:cpp-standard} (vgl. \ref{subsec:outside-lifetime})
\end{itemize}

\section{Warum wird Undefined Behavior benutzt?}
\label{sec:ub_warum}

\gls{ub} ermöglicht dem \gls{compiler} Code-Optimierungen durchzuführen, indem der \gls{compiler} davon ausgeht, dass das Programm nur definierte Operationen durchführt. \cite[S.633]{book:taming-ub}

Das Verwenden von \gls{ub} resultiert in einer Optimierung der Perfomance. Für diese wird jedoch die Vorhersehbarkeit und Sicherheit des Codes eingeschränkt. \cite[S.633]{book:taming-ub} \\
Es wird den Entwicklern überlassen das Auftreten von \gls{ub} zu verhindern oder zu minimieren.

Andere Programmiersprachen, wie Java, sollen in ihrer \gls{ir} das Auftreten von \gls{ub} verringern oder verhindern. \cite[S.633]{book:taming-ub}