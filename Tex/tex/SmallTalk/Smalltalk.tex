\chapter{Small Talk}
\label{ch:smalltalk}

\section{Was ist Small Talk?}
\label{sec:smalltalk_was}

Als Small Talk (engl. \textit{small} \glqq{}unbedeutend, klein\grqq{} und \textit{to talk} \glqq{}sich unterhalten\grqq{}) wird ein spontanes, zufällig entstandenes, lockeres Gespräch
bezeichnet. Die häufigsten Themen beziehen sich dabei meist auf das Privatleben der Involvierten, oder auf das Geschehen um das Gespräch. Der Ton des Gesprächs ist informell.\cite{misc:wikipedia_smalltalk} \newline
Im Alltag kann es in verschiedenen Situationen zu einem Small Talk kommen:
\begin{itemize}
    \item Auf dem Weg zur Arbeit
    \item In der Mittagspause
    \item Beim Einkaufen
    \item Beim Spazierengehen
    \item An der Supermarktkasse
    \item In öffentlichen Verkehrsmitteln
    \item Auf einer Party
    \item \dots
\end{itemize}

Small Talk kann in verschiedenen Situationen, beispielsweise in einem Bewerbungsgespräch, als \glqq{}Eisbrecher\grqq{} verwendet werden. Das Führen von Small Talk hilft dabei Interesse am
Gegenüber zu zeigen. In einem beruflichen Umfeld können somit Beziehungen zu Kollegen/-innen geknüpft werden. \newline
Die Auswahl der Gesprächsthemen orientiert sich dabei daran, wie gut man das Gegenüber bereits kennt. Handelt es sich um eine/-n Freund/-in, so kann der Small Talk auch übersprungen werden
und das Gespräch kann mit einem bevorzugten Thema begonnen werden. \cite{misc:wikipedia_smalltalk}

\section{Small Talk führen}
\label{sec:smalltalk_führen}

Typische Fragen im Small Talk, wie
\begin{itemize}
    \item Wie findest du das Wetter?
    \item Wie geht es dir/deiner Familie?
\end{itemize}
können als Gesprächseinstieg dienen, sind aber nicht weiterführend und können somit die Konversation zum Stehen bringen.  \newline
Offene Fragen, also Fragen welche nicht mit \glqq{}Ja\grqq{} oder \glqq{}Nein\grqq{} beantwortet werden können, erfordern eine detailliertere Antwort vom Gegenüber und fördern somit den
Gesprächsfluss. Beispiele für Anfänge einer offenen Frage sind \glqq{}Warum\grqq{} oder \glqq{}Was\grqq{}.

Gemeinsamkeiten, wie der Ort des Gespräches, stellen eine gute Grundlage für Themen dar. Durch diese wird eine vertraute Atmosphäre geschaffen. \cite[S.~104]{Birgelen:Ich-und-der-Kunde} \newline
Während des Small Talks bietet es sich an, öfters das Thema zu wechseln. Small Talk ist ein unbeschwertes Gespräch und wird durch Themenwechsel abwechslungsreich, leicht und locker. \cite[S.~108]{Birgelen:Ich-und-der-Kunde}

Neben der allgemeinen Gesprächsführung spielen, für eine gelungene Konversation, auch Mimik, Gestik und Haltung eine Rolle. \newline
Bei der Mimik ist hierbei auf Augenkontakt zu achten, schüchterne Menschen können dabei auch auf den Mund schauen. Des Weiteren haben ein Lächeln und das Vermeiden von Grimassen einen
positiven Einfluss auf das Gespräch. \newline
Das Verwenden der Arme und Hände zur Gestikulation in einem Gespräch machen dieses lebendiger. Eine belebte Körpersprache zeigt Offenheit und Extraversion. Versteckte Handflächen
und verschränkte Arme deuten hingegen auf Verschlossenheit hin. \cite[S.~119]{Birgelen:Ich-und-der-Kunde} \newline
Eine starke und selbstbewusste Haltung zieht Aufmerksamkeit auf den/die Redner/-in. Dies kann auch das Selbstbewusstsein des/der Redners/-in steigern, was schüchternen Menschen bei der
Gesprächsführung helfen kann. Zu einer selbstbewussten Körperhaltung gehören: \cite{misc:rhetorik_selbstbewusstsein}
\begin{itemize}
    \item Fester Stand: beide Beine fest und ungefähr schulterbreit auf dem Boden
    \item Aufrechte Schultern
    \item Gerader Rücken
    \item Nach vorne gerichteter Kopf, nicht nach unten schauen
\end{itemize}
Wenn das Gegnüber eine zurückhaltende Person ist, sollte eine zu selbstbewusste Haltung vermieden werden, da sich das Gegenüber sonst unter Druck gesetzt fühlen kann und in der Konversation somit
in die Defensive gerät.

%Begrüßung in drei schritten: hallo, ich bin der/die... , ich kenne den gastgeber durch... (Köder, zusätzliche information)

%für tipss: "nice to know" kann leicht gemacht werden -> bezieht sich auf Geburtstag(Geburtsstein, Sternzeichen,...), Augenfarbe(häufigkeit, Bedeutung, ...), Name(Bedeutung, Herkunft, ...) und das ganze Zeug -> nett für ein Gespräch