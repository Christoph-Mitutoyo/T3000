\chapter{Zusammenfassung}
\label{ch:zusammenfassung}

Mit dieser Arbeit sollte ein kontextabhähniger Kommunikationsassistent entwickelt werden (\ref{sec:vision}). Dies wurde in Form einer Applikation namens Strike Up umgesetzt.

Strike Up ist eine für Android entwickelte App, welche dem/der Nutzer/-in ermöglicht ein eigenes Profil, sowie Profile von Gesprächspartnern/-innen, zu erstellen. Diese Profile enthalten bevorzugte und gemiedene Tags, anhand welcher während einer Konversation Fragen ausgewählt und vorgeschlagen werden. Des Weiteren haben automatisch generierte und durch den/die Nutzer/-in ausgewählte Umgebungsvariablen einen Einfluss auf die Auswahl der Fragen. \newline
Fragen beziehen sich somit auf bevorzugte und weniger bevorzugte Themen des/der Nutzers/-in und des Gegenübers, sowie auf die Umgebung der Konversation.

Um Strike Up nutzern zu können, müssen sich Nutzer/-innen mit einer E-Mail-Adresse und einem Passwort anmelden. Anschließend wird eine Altersbestätigung, sowie eine Einwilligung zur Speicherung der Daten gefordert.

Jegliche von Strike Up zur Verfügung gestellte, sowie durch den/die Nutzer/-in erstellte, Daten werden in der, von Google gehosteten, Cloud-Datenbank Firebase gespeichert. Nutzer/-innen können dabei alle Teile der Datenbank lesen, welche keine Informationen anderer Nutzer/-innen enthalten. Schreibzugriffe werden Nutzern/-innen nur bei selbst erstellten Daten (eigenes Profil, Gesprächspartner/-innen) gestattet.

Als besonders interessant haben sich die im Zuge dieser Arbeit durchgeführten Praxistests erwiesen. \newline
Die zufällig ausgewählten Gesprächspartner/-innen reagierten meist positiv auf das Verwenden einer App als Hilfestellung und fanden dies sogar sympathisch. \newline
Während einem Gespräch hat sich dabei die Vorbereitung auf das Gespräch als am hilfreichsten erwiesen. Weniger hilfreich war die Auswahl bevorzugter Fragen. Hierbei hat sich das Ausschließen unpassender Fragen als nützlicher erwiesen.