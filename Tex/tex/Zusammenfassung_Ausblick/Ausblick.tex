\chapter{Ausblick}
\label{ch:ausblick}

Bevor Strike Up im Google Play Store, oder firmenintern, veröffentlicht werden kann, müssen noch einige Features verbessert werden: Die Benutzeroberfläche sollte ansprechender werden
und das editieren von Tags könnte, für eine bessere Übersicht, in einem (statt zwei verschiedenen) Bildschirm stattfinden. \newline
Fragen und Hinweise können noch nicht durch den/die Nutzer/-in bewertet werden und ein Feedback am Ende eines Gespräches ist auch nicht vorhanden (vgl. \ref{tab:funktional}, F-2, F-4, F-10).
Des Weiteren können Fragen noch nicht aus der Auswahl ausgeschlossen werden (F-3). Auch die Versorgung mit aktuellen Themen, beispielsweise durch einen \gls{rssfeed}, wurde nicht umgesetzt (F-8, F-9).
Es gibt auch keine einfache Funktion, mit welcher ein/-e Nutzer/-in sein/ihr Konto und damit verbundene Daten löschen kann (F-11). Dies ist möglich, indem der Verwalter der Datenbank kontaktiert wird,
jedoch ist diese Funktion nicht in Strike Up intgriert.

In der aktuellen Form kann Strike Up noch in viele Richtungen verbessert und erweitert werden: Die Gesprächsunterstützung könnte von einer Unterstützung bei One-on-One-Gesprächen auf
eine Unterstützung während Gruppengesprächen und Diskussionen erweitert werden. \newline
Das Bewerten von Fragen kann durch ein Fragen-Rating, ähnlich wie bei der Fragenbewertung vor einem Gespräch, ermöglicht werden. \newline
Nutzer/-innen könnten, durch drücken eines Buttons, die aktuell vorgeschlagene Frage in eine Liste ignorierter Fragen setzen. Vor einem Gespräch würden, für die Konversation vorgeschlagene, Fragen
mit der Liste der ignorierten Fragen verglichen werden. Wenn eine Fragen in beiden Listen vorkommt, so würde diese nicht während dem Gespräch vorgeschlagen werden. \newline
Strike Up könnte während einer Konversation Fragen als Push-Benachichtigung anzeigen. Somit müsste das Smartphone zum Einsehen einer Frage nicht mehr entsperrt werden. \newline
Die App kann um eine Vorlese-Funktion erweitert werden. Fragen werden dem/der Nutzer/-in über Kopfhörer vorgelesen. Mit den Kopfhörerfunktionen (überspringen, zurück) könnte zwischen Fragen
gewechselt werden. \newline
Des Weiteren könnte der aktuelle Standord benutzt werden um auf Attraktionen in der Nähe hinzuweisen oder um Wetterinformationen abzurufen. Die Wetterinformationen könnten dabei über
OpenWeather \cite{misc:openweather} abgerufen werden. \newline
Strike Up kann über eine zusätzliche App für Smartwatches erweitert werden. Während einem Gespräch könnten somit Fragen und Hinweise auf der Smartwatch angezeigt werden und das Smartphone
bliebe somit in der Tasche.
