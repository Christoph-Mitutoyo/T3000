\chapter{Einleitung}
\label{ch:einleitung}

Im alltäglichen Leben kommt man in verschiedenen Situationen mit gar nicht oder nur wenig bekannten Menschen in Kontakt: an der Supermarktkasse, beim Bäcker, oder man trifft eine bekannte Person in der Stadt. In solchen Situationen kann es zu einer kurzen Unterhaltung kommen: dem Small Talk. Weitere mögliche Situationen sind: Gespräche unter Schülern/Studenten, Konversationen an einem Messestand und Diskussionen nach einer Vorlesung.

Auch in der Firma führen Mitarbeiter beispielsweise an der Kaffeemaschine und in der Frühstücks- und Mittagspause kleine Unterhaltungen. Hier kennen sich die Mitarbeiter untereinander und wissen in den meisten Fällen einige Interessen des Gesprächspartners. Schwieriger wird es, wenn man das Gegenüber nicht kennt, wie zum Beispiel auf einer Messe oder einer Fortbildung. \newline
Die meisten Teilnehmer einer solchen Fortbildung kennen nur ein bis zwei weitere Teilnehmer. Um nicht durchgehend alleine zu sein muss also Kontakt zu fremden Personen aufgebaut werden. Dabei sollte auf die Interessen des Gegenübers und eventuelle Tabu- und Konfliktthemen geachtet werden.


\section{Problemstellung}
\label{sec:problemstellung}

Konkret ist, für kommunikationsschwache Menschen, in solchen Situationen der Einstieg in das Gespräch schwierig. Aus einem geringen Informationspool über das Gegenüber gilt es einen passenden Einstieg in ein Gespräch zu finden. Standardfragen, wie beispielsweise eine Bemerkung zum Wetter, können in Betracht gezogen werden, haben aber, aus gesellschaftlicher Sicht, einen Stellenwert als unkreativen Einstieg und führen nicht zwingend zu einem angeregten Gespräch.

In einem laufenden Gespräch, kann es kommunikationsschwachen Menschen aufgrund fehlender Übung an weiterführenden Fragen, Anregungen und Themen mangeln. Die Konversation gerät ins Stocken und kann abrupt abbrechen.

Das Reden mit fremden Menschen ist für Menschen mit einer Kommunikationsschwäche oft eine Stresssituation. Stress kann dazu führen, dass das Gehirn blockiert und nicht mehr leistungsfähig ist \cite[S.~79~f.]{Roese:Lernen-OHNE-Stress}. Es kommt zu einem Blackout. Gesprächsthemen und Fragen an das Gegenüber verschwinden aus dem Gedächtnis und die Konversation wird einseitig weitergeführt oder abgebrochen.

Menschliche Hilfestellungen, beispielsweise in Form eines Mentors, sind nur vor einem Gespräch verfügbar. Eine fortlaufende Unterstützung während der Konversation ist durch eine Begleitperson realisierbar, dadurch kann allerdings auch ein Dialog zwischen der Begleitperson und dem Gesprächspartner entstehen, wobei die Hilfe benötigende Person außen vor gelassen wird.

%\todo[inline]{hier steht ein todo in einer Zeile}


\section{Aufgabenstellung}
\label{sec:aufgabenstellung}

In dieser Praxisarbeit soll eine Softwarelösung entwickelt werden, welche dem Benutzer Fragen bereit stellt, um eine Konversation zu beginnen. \newline
Die Fragen sollen aufgrund von Umgebungsvariablen ausgewählt werden, damit ein passender Einstieg in ein Gespräch ermöglicht wird. Zudem können Anwender Profile von sich selbst und ihren Gegenübern erstellen. Die Einstellungen der Profile sollen auch in die Auswahl der Fragen miteinbezogen werden. \newline
Die Anwendung soll leicht zu bedienen sein und über ein übersichtliches \gls{ui} verfügen. Dies soll eine Nutzung während einer Konversation ermöglichen, ohne dass der Redefluss stark unterbrochen wird.\newline
Die Software soll ihrem Anwender im Idealfall genug theoretische Hilfestellungen bieten, damit mögliche Anwender über die Zeit eigenständig Gespräche beginnen und diese auch weiterführen können.

%Test \todo{Hier steht ein Todo mit blöder Formatierung}