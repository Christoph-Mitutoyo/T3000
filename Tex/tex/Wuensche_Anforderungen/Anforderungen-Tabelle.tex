\begin{table}[H]
    {
        \begin{tabularx}{\linewidth}{|C{0.5cm}|X|C{2.0cm}|C{2.5cm}|C{2.0cm}|}
            \hline
            ID & Kurzbeschreibung & Priorität\textsuperscript{1} & Aufwand & Kritilalität\textsuperscript{2}\\
            \hline
            \cline{1-5}
                1
                & TOOLPOS Datei öffnen, speichern, schließen
                & 1
                & sehr Niedrig
                & Keine
                \\
            \cline{1-5}
                2
                & TOOLPOS Datei bearbeiten
                & 1
                & mittel
                & Keine
                \\
            \cline{1-5}
                3
                & TOOLPOS Datei validieren
                & 1
                & Mittel
                & Mittel
                \\
            \cline{1-5}
                4
                & TOOLPOS Datei schneller identifizieren
                & 3
                & Niedrig
                & Keine
                \\
            \cline{1-5}
                5
                & Verfahrwege aus TOOLPOS Datei anzeigen
                & 1
                & Hoch
                & Mittel
                \\
            \cline{1-5}
                6
                & Zwei Verfahrwege gleichzeitig anzeigen
                & 4
                & Mittel
                & Niedrig
                \\
            \cline{1-5}
                7
                & Anzeige bei Änderungen aktualisieren
                & 2
                & Mittel
                & Mittel
                \\
            \cline{1-5}
                8
                & Kollisionswarnung
                & 5
                & Sehr Hoch
                & Mittel
                \\
            \hline
        \end{tabularx}
    }
    \caption{Anforderungen}
    \textsuperscript{1} \quad Je niedriger die Zahl, desto höher die Priorität \\
    \textsuperscript{2} \quad Für Definition siehe \cite[]{Balzert:Lehrbuch-der-softwaretechnik}
    \label{tab:anforderungen1}
\end{table}
