\chapter{Requirements Engineering}
\label{ch:requirements}

Das Requirements Engineering bezieht sich in diesem Projekt auf das Erfassen von Stakeholdern und das Ermitteln von Anforderungen.

Wegen fehlenden oder falschen Anforderungen können viele Projekte nicht die gewünschten Ziele erreichen \cite[S.~4-9]{Ebert:SystematischesReqEng}. Aus diesem Grund ist es sinnvoll zum Beginn eines Projektes Stakeholder zu erfassen, eine Vision aufzustellen und Anforderungen an des Endprodukt zu ermitteln. \newline


\section{Stakeholder}
\label{sec:stakeholder}

Die Stakeholder-Analyse identifiziert wichtige Anspruchsträger des Projektes. Die nachfolge Tabelle wurde nach \cite[S.~58-60]{Ebert:SystematischesReqEng} ausgefüllt. Die Spalte \glqq{}Verantwortung\grqq{} wurde mit der Spalte \glqq{}Berechtigte Interessen\grqq{} ersetzt. Die Spalte \glqq{}Aufgabenbeschreibung\grqq{} wurde ausgelassen, da sie in \glqq{}Berechtigte Interessen\grqq{} indirekt enthalten ist und somit keine weiteren Informationen für dieses Projekt bereitstellt.

\begin{table}[H]
    {
        \tiny
        \begin{tabularx}{\linewidth}{|X|X|X|X|X|}
            \hline
            Name & Rolle                                                                                         & Berechtigte Interessen & Konfliktpotential & Verfügbarkeit \\
            \hline
            \cline{1-5}
            TW
                 & Trainer (Studenten) \newline
            Trainer (Mitarbeiter)
                 & Projekt entspricht dem Kenntnisstand des Studenten \newline
            Projekt soll innerhalb der zeitlichen Vorgaben der DHBW zu einem sichtbaren Ergebnis kommen
                 & Kann als Trainer (Vorgesetzter) Entscheidungen treffen. Gegebenenfalls auch gegen das Projekt
                 & Direkt (Stakeholder arbeitet in der gleichen Firma/Abteilung)
            \\
            \cline{1-5}
            KS
                 & Manager \newline
            Verantwortlich für die Ausbildung der Studenten
                 & Student hat Zugang zu den nötigen Mitteln für das Projekt
            Projekt wird zeitgerecht fertiggestellt
                 & Zeitvorstellung/-einteilung das Projekts \newline
            Besitzt als Vorgesetzter und Manager von TW die größte Entscheidungskraft
                 & Direkt
            \\
            \cline{1-5}
            DB
                 & Auftraggeber
                 & Projekt enthält Anforderungen der ursprünglichen Idee
                 & Anforderungen der ursprünglichen Idee können neuen Anforderungen gegenüber stehen
                 & Direkt
            \\
            \cline{1-5}
            DB, KS, TW
                 & Gutachter
                 & Dokumentation muss nach Vorgaben der DHBW erstellt sein \newline
            Faire und nachvollziehbare Bewertung \newline
            Dokumentation muss nach Vorgaben von Mitutoyo CTL erstellt sein
                 & -
                 & Direkt
            \\
            \cline{1-5}
            Prüfungsausschuss
                 & Zweitgutachter
                 & Dokumentation entspricht den Anforderungen der DHBW
                 & Konfliktpotential mit Erstgutachtern
                 & -
            \\
            \cline{1-5}
            Mitarbeiter
                 & Endkunden
                 & Die Softwarelösung soll benutzbar sein \newline
            Die Softwarelösung erfüllt ihren Zweck
                 & -
                 & Direkt
            \\
            \cline{1-5}
            RE
                 & Datenschutzbeauftragter
                 & Die Softwarelösung entspricht der DSGVO
                 & Datenverwaltung
                 & Direkt
            \\
            \cline{1-5}
            PK
                 & Geschäftsführer
                 & Projekt wird frisgerecht abgeschlossen
                 & Ressourcenverteilung (Zeit, Verfügbarkeit der Trainer, ...)
                 & Direkt
            \\
            \cline{1-5}
            CB
                 & Entwickler
                 & Erfolg des Projeks \newline
            Lauffähige Anwendung
                 & -
                 & -
            \\
            \hline
        \end{tabularx}
    }
    \caption{Stakeholder}
    \label{tab:stakeholder}
\end{table}


\section{Vision}
\label{sec:vision}

Vor der Festlegung von Anforderungen ist es notwendig die Vision und Ziele zu definieren. Dies ermöglicht einen Abgleich der Anforderungen mit der Vision und den Zielen. Daraus kann entschieden werden, ob die Anforderung zielführend ist. \cite[S.~456~ff.]{Balzert:Lehrbuch-der-softwaretechnik}

Vision:
\par
\begingroup
\leftskip=30pt
\noindent
Die Softwarelösung bringt Menschen mit Kommunikationsschwäche dazu, den ersten Schritt in ein Gespräch zu wagen. Dazu schlägt sie Sätze vor, die in der Situation des Benutzers als geeignet erscheinen, ein längeres Gespräch mit einem bestimmten Ziel zu verfolgen.

Für die Softwarelösung wird im weiteren Verlauf der Name \glqq{}Strike Up\grqq{} (englisch: \glqq{}to strike sth. up\grqq{} = \glqq{}an etwas anknüpfen\grqq{}) verwendet.

Strike Up unterscheidet sich insbesondere dadurch von anderen Small-Talk-Ratschlägen, dass nicht nur plumpe Floskeln, sondern maßgeschneiderte persönliche Formulierungen angeboten werden.

Zudem ist Strike Up erweiterbar und kann um neue Formulierungen und neue Situationen ergänzt werden.
\par
\endgroup
Ziele: \newline
\begin{itemize}
    \item Es werden Sätze und Fragen zur Eröffnung eines Dialogs bereitgestellt.
    \item Sätze und Fragen werden auf den Anwender angepasst.
    \item Sätze und Fragen werden auf das Gegenüber angepasst.
    \item Sätze und Fragen werden auf das Ziel und die Umgebung der Unterhaltung angepasst.
    \item Strike Up lässt sich leicht bedienen.
\end{itemize}


\section{Anforderungen}
\label{sec:anforderungen}

Durch die Befragung mehrerer Stakeholder wurden Anforderungen an das Projekt ermittelt.

Anforderungen erhalten die Anforderungsattribute Identifikator, Beschreibung, Priorität und Aufwand \cite[S.~479~f.]{Balzert:Lehrbuch-der-softwaretechnik}. Da die in diesem Projekt enwickelte Softwarelösung keinen direkten Schaden an Menschen verursachen kann, wird das Anforderungsattribut \glqq{}Kritikalität\grqq{} außer Acht gelassen.

\subsection{Funktionale Anforderungen}
\label{subsec:funktional}

\begin{table}[H]
    {
        \begin{tabularx}{\linewidth}{|C{1cm}|X|C{2cm}|}
            \hline
            ID & Beschreibung                                                                                                & Stakeholder \\
            \hline
            \cline{1-3}
            F-1
               & Es werden allgemeine Hinweise für eine bessere Gesprächsführung bereitgestellt.
               & KS
            \\
            \cline{1-3}
            F-2
               & Strike Up besitzt eine Feedbackfunktion, mit welcher Hinweise als gut oder schlecht bewertet werden können.
               & KS, DB, TW
            \\
            \cline{1-3}
            F-3
               & Als schlecht empfundene Hinweise werden kein zweites Mal vorgeschlagen.
               & KS
            \\
            \cline{1-3}
            F-4
               & Es werden Statistiken für die Nützlichkeit der Hinweise bereitgestellt.
               & KS
            \\
            \cline{1-3}
            F-5
               & Es werden mehrere Fragen/Hinweise gleichzeitig bereitgestellt.
               & DB, KS, TW
            \\
            \cline{1-3}
            F-6
               & Es lassen sich Personenprofile erstellen (von sich selbst und vom Gegenüber)
               & DB
            \\
            \cline{1-3}
            F-7
               & Personenprofile können personalisierte Notizen enthalten, welche in einem Gespräch abrufbar sind.
               & KS
            \\
            \cline{1-3}
            F-8
               & Es werden aktuelle Themen angeboten, über welche der Benutzer Gespächsstoff beziehen kann.
               & KS
            \\
            \cline{1-3}
            F-9
               & Es kann ausgewählt werden von welchen Quellen die Themen stammen.
               & KS
            \\
            \cline{1-3}
            F-10
               & Nach einem Gespräch wird dem Benutzer angeboten Feedback zu senden. Dies geschieht über einen Button.
               & TW, KS
            \\
            \cline{1-3}
            F-11
               & Benutzer können ihr Profil und damit verbundene Daten löschen.
               & RE
            \\
            \cline{1-3}
            F-12
               & Fragen/Hinweise werden an den Benutzer und das Gegenüber angepasst.
               & -
            \\
            \cline{1-3}
            F-13
               & Fragen/Hinweise werden an die Umgebung des Gesprächs angepasst.
               & -
            \\
            \cline{1-3}
            F-14
               & Im Gesprächsverlauf werden weiterführende Fragen/Hinweise bereitgestellt.
               & -
            \\
            \hline
        \end{tabularx}
    }
    \caption{Funktionale Anforderungen}
    \label{tab:funktional}
\end{table}

\subsection{Nichtfunktionale Anforderungen}
\label{subsec:nichtfunktional}

\begin{table}[H]
    {
        \begin{tabularx}{\linewidth}{|C{1cm}|X|C{2cm}|}
            \hline
            ID & Beschreibung                                                                       & Stakeholder \\
            \hline
            \cline{1-3}
            NF-1
               & Strike Up ist portabel.
               & TW, KS, DB
            \\
            \cline{1-3}
            NF-2
               & Strike Up ist unauffällig und kann leicht während einem Gespräch verwendet werden.
               & KS, DB, TW
            \\
            \cline{1-3}
            NF-3
               & Die Benutzeroberfläche der Softwarelösung ist übersichtlich.
               & DB, CB
            \\
            \cline{1-3}
            NF-4
               & Personenbezogene Daten werden nicht an Dritte weitergegeben.
               & RE
            \\
            \hline
        \end{tabularx}
    }
    \caption{Nichtfunktionale Anforderungen}
    \label{tab:nichtfunktional}
\end{table}


\subsection{Priorität der Anforderungen}
\label{subsec:anforderungen2}

\begin{table}[H]
    {
        \begin{tabularx}{\linewidth}{|X|X|}
            \hline
            Priorität & IDs                                                     \\
            \hline
            \cline{1-2}
            Niedrig (optional)
                      & F-4, F-9
            \\
            \cline{1-2}
            Mittel
                      & F-2, F-5, F-7, F-8, F-10, F-11
            \\
            \cline{1-2}
            Hoch
                      & NF-1, NF-2, F-1, F-3, F-6, NF-3, NF-4, F-12, F-13, F-14
            \\
            \hline
        \end{tabularx}
    }
    \caption{Priorität der Anforderungen}
    \label{tab:anforderungen2}
\end{table}

\subsection{Aufwandsschätzung der Anforderungen}
\label{subsec:anforderungen3}

\begin{table}[H]
    {
        \begin{tabularx}{\linewidth}{|X|X|}
            \hline
            Aufwand & IDs                                             \\
            \hline
            \cline{1-2}
            Niedrig
                    & NF-1, F-5, NF-4
            \\
            \cline{1-2}
            Mittel
                    & NF-2, F-1, F-3, F-7, F-8, F-9, F-10, F-11, F-14
            \\
            \cline{1-2}
            Hoch
                    & F-2, F-4, F-6, NF-3, F-12, F-13
            \\
            \hline
        \end{tabularx}
    }
    \caption{Aufwandsschätzung der Anforderungen}
    \label{tab:anforderungen3}
\end{table}

\subsection{User Stories}
\label{subsec:userstories}

Chris Rupp verweist beim Requirementsengineering auf verschiedene Spezifikationslevel. Diese beschreiben die genaue Spezifikation einer Anforderung. \newline
Eine Anforderung des Spezifikationslevels Null entspricht dabei einer Idee, einer vagen Formulierung oder einem allgemeinen Ziel. Eine Anforderung auf dem vierten Spezifikationslevel entspricht einer Technical \gls{userstory} oder einem Backlog Item. \cite[S.~38~ff.]{Rupp:ReqEng}

Für dieses Projekt ist eine genaue Spezifikation der Anforderung auf das Level drei und vier nicht sinnvoll. Jedoch befinden sich die bisherigen Anforderungen auf dem nullten und ersten Spezifikationslevel. Um Anforderungen des zweiten Spezifikationslevels zu generieren werden nun, mit den in \ref{sec:personas} erstellten Personas, \glspl{userstory} geschaffen.

\pagebreak
\begin{table}[H]
    {
        \begin{tabularx}{\linewidth}{|X|C{2,5cm}|}
            \hline
            Beschreibung & Anforderung \\
            \hline
            \cline{1-2}
            Als Daniel Dacher möchte ich, dass meine an Strike Up gesendeten Daten nicht an Dritte weitergegeben werden.
                         & NF-4
            \\
            \cline{1-2}
            Als Daniel Dacher möchte ich, dass ich mit einem Klick mein Konto und alle damit verbundenen Daten jederzeit löschen kann.
                         & F-11
            \\
            \cline{1-2}
            Als Susanne Schmid möchte ich, dass ich Strike Up leicht während der Fahrt mit Bus und Bahn verwenden kann.
                         & NF-1, NF-2
            \\
            \cline{1-2}
            Als Susanne Schmid möchte ich, dass ich in einem Gespräch schnell auf die Funktionen von Strike Up zugreifen kann.
                         & NF-2
            \\
            \cline{1-2}
            Als Susanne Schmid möchte ich, dass ich keine schlechten Fragen/Hinweise bekomme, da meine Gesprächszeit oft beschränkt ist.
                         & F-3, F-12
            \\
            \cline{1-2}
            Als Susanne Schmid möchte ich eine Möglichkeit haben, mit welcher ich mich im Voraus auf ein Gespräch vorbereiten kann.
                         & F-1
            \\
            \cline{1-2}
            Als Torsten Tacheles, möchte ich durch Strike Up neue Kommunikationstechniken lernen.
                         & F-1
            \\
            \cline{1-2}
            Als Torsten Tacheles möchte ich durch Strike Up täglich etwas Neues lernen, was ich meinen Klienten beibringen kann.
                         & F-1, F-8
            \\
            \cline{1-2}
            Als Torsten Tacheles möchte ich, dass Strike Up auch meinen Klienten weiterhilft.
                         & -
            \\
            \hline
        \end{tabularx}
    }
    \caption{User Stories von Daniel Dacher, Susanne Schmid und Torsten Tacheles}
    \label{tab:userstories}
\end{table}

