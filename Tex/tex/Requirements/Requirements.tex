\chapter{Requirements Engineering}
\label{ch:requirements}

Das Requirements Engineering bezieht sich in diesem Projekt auf das Erfassen von Stakeholdern und das Ermitteln von Anforderungen.

Wegen fehlenden oder falschen Anforderungen können viele Projekte nicht die gewünschten Ziele
erreichen \cite[S.~4-9]{Ebert:SystematischesReqEng}. Aus diesem Grund ist es sinnvoll zum Beginn eines Projektes Stakeholder zu erfassen,
eine Vision aufzustellen und Anforderungen an das Endprodukt zu ermitteln.


\section{Stakeholder}
\label{sec:stakeholder}

Die Stakeholder-Analyse identifiziert wichtige Anspruchsträger des Projektes.
Die nachfolge Tabelle wurde nach \cite[S.~58-60]{Ebert:SystematischesReqEng} ausgefüllt.
Eine Analyse der Verantwortung einzelner Stakeholder entfällt, da sich diese in dieser Arbeit mit der Rolle deckt. Zusätzlich arbeitet jeder
Stakeholder in der gleichen Firma/Abteilung, weshalb die Spalte \glqq{}Verfügbarkeit\grqq{} ausgelassen wird.

\begin{tiny}
\begin{longtable}[H]{llL{3cm}L{3cm}L{3cm}}
            \toprule
            \thead[l]{Rolle}            & 
            \thead[l]{Name}                               & 
            \thead[l]{Aufgabenbeschreibung}                                             & 
            \thead[l]{Hintergrundinformationen}                                                          & \thead[l]{Konfliktpotenzial}                                              \\
            \midrule
            \endhead
            Gutachter Intern & 
            T\censor{homas} W\censor{eller}    & 
            Aus- und Weiterbildung des Personals                             & 
            Betreut/bewertet die Studienarbeit \newline Beratung/Analyse von \ref{sec:windbg} &                                                                \\
            \midrule
            Gutachter DHBW   &                                    
            & 
            Gutachter                                                        & 
            Bewertet die Studienarbeit                                                        &                                                                \\
            \midrule
            Entwickler       &                                    
            & Softwareentwicklung                                              & 
            Soll Tools benutzen um \gls{ub} zu vermeiden                                      &                                                                \\
            \midrule
            Tester           &                                    
            & 
            Testen der Software                                              & 
            Weniger Fehler in der Software erleichtern die Arbeit                             &                                                                \\
            \midrule
            Architekt        & 
            D\censor{ieter} B\censor{locher}   & 
            Kontrollieren des Source Code falls Selbstkontrolle nicht klappt & 
            Hat den Bug bearbeitet und behoben                                                & Konflikt mit bisher verwendetem/vorgeschlagenem Tool           \\
            \midrule
            Build Manager    & 
            T\censor{obias} F\censor{riedrich} & DevOps                                                           & 
            Verantwortlich für TFS Zugang und Build Prozesse                                  &                                                                \\
            \midrule
            Geschäftsführer  & 
            P\censor{eter} K\censor{lein}      & 
            Ressourcenverteilung (Zeit, Verfügbarkeit der Trainer, \dots)    & 
            Finanziert Studium \newline Finanziert Tool                                       & Zeitliche Verfügbarkeit einer Lösung \newline Kosten des Tools \\
            \midrule
            Student          & 
            Christoph Böhringer                & 
            Verfasst die Arbeit                                              &                                                                                   &                                                                \\
            \bottomrule
    \caption{Stakeholder}
    \label{tab:stakeholder}
\end{longtable}
\end{tiny}

\section{Anforderungen}
\label{sec:anforderungen}

Durch die Befragung der Stakeholder wurden Anforderungen an das Projekt ermittelt.

Anforderungen erhalten die Anforderungsattribute Identifikator, Beschreibung und Priorität.
\cite[S.~479~f.]{Balzert:Lehrbuch-der-softwaretechnik}. Da die in diesem Projekt keine Software entwickelt werden soll, wird das Anforderungsattribut \glqq{}Aufwand\grqq{} außer
Acht gelassen. Es kann auch kein direkter Schaden an Menschen verursacht werden, weshalb eine Bewertung der \glqq{}Kritikalität\grqq{} entfällt.

\subsection{Funktionale Anforderungen}
\label{subsec:funktional}

\begin{table}[H]
    {
        \begin{tabularx}{\linewidth}{|p{1cm}|X|p{2cm}|}
            \hline
            ID  & Beschreibung                                                                               & Stakeholder              \\
            \hline
            \cline{1-3}
            F-1 & Das Tool soll nicht initialisierte Variablen hervorheben                                   & TW, DB, \ref{sec:windbg} \\
            \cline{1-3}
            F-2 & Das Tool soll Nullpointer Dereferenzierung hervorheben                                     & TW                       \\
            \cline{1-3}
            F-3 & Das Tool soll keinen/möglichst wenig korrekten Code hervorheben                            & DB, TF                   \\
            \cline{1-3}
            F-4 & Das Tool soll über eine config oder cmd Datei automatisierbar sein                         & DB, TF                   \\
            \cline{1-3}
            F-5 & Das Tool soll erklären warum eine Stelle hervorgehoben wurde und Lösungsvorschläge liefern & DB                       \\
            \cline{1-3}
            F-6 & Das Tool soll eine Risikoanalyse der markierten Fehler durchführen                         & DB                       \\
            \hline
        \end{tabularx}
    }
    \caption{Funktionale Anforderungen}
    \label{tab:funktional}
\end{table}

\subsection{Nichtfunktionale Anforderungen}
\label{subsec:nichtfunktional}

\begin{table}[H]
    {
        \begin{tabularx}{\linewidth}{|p{1cm}|X|p{2cm}|}
            \hline
            ID   & Beschreibung                                                                   & Stakeholder \\
            \hline
            \cline{1-3}
            NF-1 & Das Tool ist mit anderen, bereits im Unternehmen verwendeten, Tools kompatibel & TF, DB      \\
            \cline{1-3}
            NF-2 & Das Tool liegt innerhalb der Kostenvorstellung                                 & PK          \\
            \cline{1-3}
            NF-3 & Für das Tool ist eine Anwendungsanleitung verfügbar                            & DB          \\
            \hline
        \end{tabularx}
    }
    \caption{Nichtfunktionale Anforderungen}
    \label{tab:nichtfunktional}
\end{table}