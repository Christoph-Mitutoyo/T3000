\chapter{Requirements Engineering}
\label{ch:requirements}

Das Requirements Engineering bezieht sich in diesem Projekt auf das Erfassen von Stakeholdern und das Ermitteln von Anforderungen.

Wegen fehlenden oder falschen Anforderungen können viele Projekte nicht die gewünschten Ziele
erreichen \cite[S.~4-9]{Ebert:SystematischesReqEng}. Aus diesem Grund ist es sinnvoll zum Beginn eines Projektes Stakeholder zu erfassen,
eine Vision aufzustellen und Anforderungen an des Endprodukt zu ermitteln.


\section{Stakeholder}
\label{sec:stakeholder}

Die Stakeholder-Analyse identifiziert wichtige Anspruchsträger des Projektes.
Die nachfolge Tabelle wurde nach \cite[S.~58-60]{Ebert:SystematischesReqEng} ausgefüllt.
Eine Analyse der Konfliktpotienzale entfällt, da es sich bei diesem Projekt um eine studentische Arbeit handelt, wobei keine Konflikte bei
der Entwicklung entstehen.

\begin{table}[H]
    {
        \tiny
        \begin{tabularx}{\linewidth}{|X|X|X|X|X|X|}
            \hline
            Rolle
             & Name
             & Aufgabenbeschreibung
             & Verantwortung
             & Hintergrundinformationen
             & Verfügbarkeit
            \\
            \hline
            \cline{1-6}
            Gutachter Intern
             & TW
             & Aus- und Weiterbildung des Personals
             & Trainer (Studenten) \newline
            Trainer (Mitarbeiter)
             & Betreut die Studienarbeit
             & Direkt (Stakeholder arbeitet in der gleichen Firma/Abteilung)
            \\
            \cline{1-6}
            Entwickler
             &
             & Softwareentwicklung
             &
             & Soll Tools benutzen um \gls{ub} zu vermeiden
             &
            \\
            \cline{1-6}
            Tester
             &
             & Testen der Software
             & Tester
             & Weniger Fehler in der Software erleichtert die Arbeit
             &
            \\
            \cline{1-6}
            Architekt
             & DB
             & Kontrollieren des Source Code falls Selbstkontrolle nicht klappt
             & Softwarearchitekt
             & Hat den Bug bearbeitet und behoben
             & Direkt
            \\
            \cline{1-6}
            -
             & TF
             &
             &
             & Verantwortlich für TFS Zugang
             & Direkt
            \\
            \cline{1-6}
            Geschäftsführer
             & PK
             & Ressourcenverteilung (Zeit, Verfügbarkeit der Trainer, \dots)
             & Geschäftsführer
             & Finanziert Studium
             & Direkt
            \\
            \hline
        \end{tabularx}
    }
    \caption{Stakeholder}
    \label{tab:stakeholder}
\end{table}