\chapter{Praxistests}
\label{ch.praxistests}

Strike Up soll Menschen beim Eröffnen und Führen von Gesprächen helfen (\ref{sec:vision}). Um die Wirksamkeit diser Hilfestellungen zu testen, wurden mit der entwickelten Applikation Praxistests durchgeführt. \newline
Gesprächspartner/-innen waren dabei zufällige Passanten auf der Straße und in Drogeriemärkten. \newline
Die folgenden Erkenntnisse beruhen auf empirischen Erfahrungen.

Während den Tests erwies sich besonders das Vorbereiten auf das Gespräch als hilfreich. Präparierte Gesprächsthemen halfen dabei, eine Konversation interessant zu gestalten und am Laufen zu halten. \newline
Da die Versorgung mit Gesprächsthemen durch Strike Up aus zeitlichen Gründen in dieser Arbeit nicht zureichend entwickelt wurde, wurden externe Quellen (News-Webseiten, Nachrichten im Fernesehen) hinzugezogen. Mit diesen Quellen
wurde das von Strike Up gewünschte Verhalten nach F-9 und F-10 simuliert.

Auf sich selbst und das Gegenüber angepasste Fragen erwiesen sich als vergleichsweise durchschnittlich nützlich. Fragen entstanden eher spontan und aus den bereits vorbereiteten Themn heraus. \newline
Das Vermeiden unpassender Fragen hingegen sorgte dafür, dass für sich selbst und das Gegenüber unangenehme Themen nicht angesprochen wurden. Dies verhinderte einen abrupten Abbruch der Konversation.

Eine unerwartete Entdeckung war, dass das Gegenüber, wenn es die Zuhilfenahme der App bemerkte, meist positiv mit einem Lachen reagierte. Als Grund für das Lachen wurde genannt, dass das Verwenden der App den/die Benutzer/-in sympathisch darstelle. Der/die Nutzer/-in würde eine Schwäche einsehen und versuche diese zu verbessern. \newline
Es kamen aber auch Fälle vor, bei welchen sich das Gegenüber ablenend gegenüber der App äußerte und das Gespräch somit beendete. Das Smartphone würde dabei dem/der Nutzer/-in die Konversationsführung wegnehmen und somit zu einer Verdummung führen.

Es wurde aber auch festgestellt, dass Strike Up um eine Schnellgespräch-Funktion erweitert werden sollte. \newline
Will der/die Nutzer/-in spontan ein Gespräch beginnen, so muss zuerst ein/-e Gesprächspartner/-in ausgewählt werden und anschließend Umgebungsvariablen gesetzt werden. Dieser Prozess kann, besonders beim Erstellen eines/-r neuen Gesprächspartners/-in Zeit in Anspruch nehmen. \newline
Um diesen Prozess zu umgehen, kann im Hauptbildschirm ein Button hinzugefügt werden, welcher sofort ein Gespräch startet. Die \gls{ui} des Gespräches entspricht dabei der \gls{ui} einer normalen Konversation. Das Profil des/der Gesprächspartners/-in wird während der Konversation entdeckt/erstellt.